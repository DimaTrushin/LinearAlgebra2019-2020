\ProvidesFile{lecture10.tex}[Лекция 10]

\newpage
\section{Векторные пространства}

\subsection{Идея и определение}

\paragraph{Идея}
Мы с вами до этого изучали много разных объектов, которые не сильно похожи друг на друга. Например, вектор-столбцы $F^n$, матрицы $\operatorname{M}_{m\,n}(F)$, функции $f\colon X\to F$, многочлены $F[x]$. Все эти товарищи нам постоянно встречаются и каждый раз приходится для каждого из них все доказывать заново и во время доказательств мы видим, что наши рассуждения повторяются. Это означает, что на самом деле у всех этих объектов есть некий общий интерфейс, через который мы на самом деле с ним работаем. Самое главное в этом интерфейсе то, что мы можем брать элементы из этих объектов, умножать эти элементы на числа и складывать между собой. Абстрактное векторное пространство как раз и формализует идею такого общего интерфейса, через который в множестве можно складывать элементы и умножать на числа. 

У такого подхода есть несколько плюсов. Во-первых, формальное удобство, как только вы что-то сделали для абстрактного векторного пространства и увидели, что что-то конкретное является таковым, то все ваши достижения автоматом применимы в этой конкретной ситуации. Общий алгоритм для векторного пространства будет одинаково хорошо работать и для столбцов, и для матриц, и для функций и т.д. Во-вторых, есть менее очевидный бонус. Когда мы доказываем что-то про абстрактное векторное пространство, то про него надо думать как про $F^n$. Это поможет вам не потеряться в формализме и догадаться, что откуда берется. Неформально это означает, что если вы что-то умеет делать для $F^n$, то это автоматически верно для любого векторного пространства! Формально это не совсем правда, но в классе хороших пространств это так.\footnote{Под хорошими тут подразумеваются конечно мерные.} Тем не менее, даже в классе всех пространств, интуиция из $F^n$ очень полезна.


\paragraph{Определение}

\begin{definition}\label{def::VectorSpace}
Пусть $F$ -- некоторое фиксированное поле. Тогда векторное пространство над полем $F$ -- это следующий набор данных $(V, +, \cdot)$, где
\begin{itemize}
\item $V$ -- множество. Элементы этого множества будут называться векторами.
\item $+\colon V \times V \to V$ -- бинарная операция, то есть правило действующее так $(v,u)\mapsto v + u$, где $u,v \in V$. 
\item $\cdot \colon F \times V \to V$ -- бинарная операция , то есть правило действующее так $(\alpha, v)\mapsto \alpha v$, где $\alpha \in F$ и $v\in V$.
\end{itemize}
При этом эти данные удовлетворяют следующим $8$ аксиомам:
\begin{enumerate}

\item {\bf Ассоциативность сложения} Для любых векторов $u,v,w\in V$ верно $(u+v) + w = u + (v+w)$.

\item {\bf Существование нулевого вектора} Существует такой вектор $0\in V$, что для любого $v\in V$ выполнено $0 + v = v + 0 = v$.

\item {\bf Существование противоположного вектора} Для любого вектора $v\in V$ существует вектор $-v\in V$ такой, что $v + (-v) = (-v) + v = 0$.

\item {\bf Коммутативность сложения} Для любых векторов $u,v \in V$ верно $u + v = v + u$.

\item {\bf Согласованность умножения со сложением векторов} Для любого числа $\alpha \in F$ и любых векторов $u,v \in V$ верно $\alpha(v + u) = \alpha v + \alpha u$.

\item {\bf Согласованность умножения со сложением чисел} Для любых чисел $\alpha, \beta\in F$ и любого вектора $v\in V$ верно $(\alpha + \beta)v = \alpha v + \beta v$.

\item {\bf Согласованность умножения с умножением чисел} Для любых чисел $\alpha,\beta\in F$ и любого вектора $v\in V$ верно $(\alpha\beta)v = \alpha(\beta v)$.

\item {\bf Нетривиальность} Для любого $v\in V$ верно $1 v = v$.\footnote{Здесь $1\in F$.}
\end{enumerate}
\end{definition}

\paragraph{Примеры}
\begin{enumerate}
\item Поле $F$ (или кто больше привык к вещественным числам $\mathbb R$) является векторным пространством над $F$ (соответственно над $R$).

\item Более обще, множество вектор-столбцов $F^n$ является векторным пространством над $F$.

\item Множество матриц $\operatorname{M}_{m\,n}(F)$ является векторным пространством над $F$.

\item Пусть $X$ -- произвольное множество, тогда множество функций $\{f\colon X\to F\}$ является векторным пространством над $F$. Надо лишь объяснить как складывать функции и умножать на элементы $F$. Операции поточечные, пусть $f,g\colon X\to F$, тогда функция $(f+g)\colon X\to F$ действует по правилу $(f+g)(x) = f(x) + g(x)$. Если $\alpha \in F$, то функция $(\alpha f)\colon X\to F$ действует по правилу $(\alpha f)(x) = \alpha f(x)$.

\item Множество многочленов $F[x] = \{a_0+a_1x + \ldots + a_n x^n\mid a_i \in F\}$. Тут надо обратить внимание, что мы подразумеваем под многочленом. Для нас многочлен -- это НЕ функция, многочлен -- это картинка вида $a_0 + a_1 x + \ldots + a_n x^n$.\footnote{Для любителей формализма, можете считать, что многочлен -- это конечная последовательность элементов $F$ вида $(a_0,\ldots,a_n)$.} Складываются и умножаются эти картинки по одинаковым правилам. Важно, что две такие картинки равны тогда и только тогда, когда у них равные коэффициенты. Множество всех многочленов $F[x]$ является векторным пространством над $F$.
\end{enumerate}

\paragraph{Замечание}
Стоит отметить, что в обычных векторных пространствах мы привыкли к некоторым свойствам, которые бы хотелось иметь и в общем случае. Например, в $F^n$ есть единственный нулевой вектор, а аксиомы в общем случае говорят, что нулевой вектор лишь существует. Однако, можно показать, что нулевой вектора автоматически единственный. Давайте перечислим некоторые непосредственные следствия из аксиом, которые я оставляю в качестве упражнения:
\begin{enumerate}
\item Нулевой вектор единственный.

\item Для любого $v\in V$ существует единственный $-v$.

\item Для любого вектора $v\in V$ верно $-v = (-1)v$.

\item Для любого вектора $v\in V$ имеем $0 v = 0$.

\item Для любого числа $\alpha \in F$ верно $\alpha 0 = 0$.
\end{enumerate}

\subsection{Подпространство}

Пусть $V$ -- векторное пространство над $F$. Тогда непустое подмножество $U\subseteq V$ называется подпространством, если на него можно ограничить операции $+$ и $\cdot$ и относительно них оно является векторным пространством. Формально, это означает, что для любых векторов $u,u'\in U$ верно, что $u+u'\in U$ и для любого скаляра $\alpha\in F$ верно, что $\alpha u \in U$. Тогда на $U$ это задает операции сложения и умножения на скаляр и надо, чтобы выполнялись все аксиомы векторного пространства для них. Оказывается, что все аксиомы будут выполняться автоматически! Например, почему у нас будет $0\in U$. Потому что если мы возьмем любой вектор $u\in U$, то $0 = 0 u \in U$. Остальное я оставлю в качестве упражнения.  

\paragraph{Примеры}
\begin{enumerate}
\item Для любого векторного пространства $V$ подмножества $0$ и $V$ всегда являются подпространствами. 

\item Множество $\{y\in F^n \mid Ay = 0\}\subseteq F^n$, где $A\in\operatorname{M}_{m\,n}(F)$, является векторным подпространством в $F^n$.
\end{enumerate}

\subsection{Линейные комбинации}

\paragraph{Мотивация}
Пусть у нас есть векторное пространство $V$ над полем $F$. Давайте поймем, а что вообще с ним можно делать? Во-первых, $V$ -- это множество. Значит из него можно брать элементы. Во-вторых, там есть операция умножения на числа, то есть любой вектор можно умножить на какое-то число. В-третьих, у нас вектора можно складывать. Все это означает, что все что можно делать с векторным пространством, это набрать каких-то векторов из него $v_1,\ldots,v_n$ и написать выражение вида $\alpha_1 v_1 + \ldots + \alpha_n v_n$, для произвольных $\alpha_i\in F$. Это выражение будет задавать нам какой-то вектор из $V$. Как мы видим, особенно не разбежишься с разнообразием действий. Однако, важно, что с помощью подобных выражений можно вытащить абсолютно всю информацию из векторных пространств, которую только возможно. 

\paragraph{Линейные комбинации}

Пусть $V$ -- некоторое векторное пространство над полем $F$ и пусть $v_1,\ldots,v_n\in V$ -- некоторый набор векторов. Тогда выражение вида $\alpha_1 v_1 +\ldots + \alpha_n v_n$, где $\alpha_i\in F$, называется линейной комбинацией $v_1,\ldots,v_n$. Линейная комбинация называется тривиальной, если все $\alpha_i = 0$. В противном случае она называется нетривиальной.

Вектора $v_1,\ldots,v_n\in V$ называются линейно зависимыми, если существует их нетривиальная линейная комбинация равная нулю, то есть для каких-то $\alpha_i\in F$ (так что хотя бы один не равен нулю) выражение $\alpha_1 v_1+\ldots + \alpha_n v_n = 0$. Подчеркнем, что вектора линейно независимы, если из равенство $\alpha_1 v_1 + \ldots + \alpha_n v_n = 0$ следует, что все $\alpha_i = 0$.

\paragraph{Примеры}
\begin{enumerate}
\item Вектор $0$ всегда линейно зависим.

\item Вектор $v\in V$ линейно зависим тогда и только тогда, когда он равен нулю.

\item Вектора $v_1, v_2 \in V$ линейно зависимы тогда и только тогда, когда они пропорциональны (то есть один из них равен другому умноженному на элемент поля).
\end{enumerate}

\paragraph{Линейная оболочка}

Пусть $E\subseteq V$ -- некоторое подмножество в векторном пространстве $V$ над полем $F$. Тогда обозначим через $\langle E \rangle$ множество всех линейных комбинаций векторов из $E$, то есть
\[
\langle E \rangle = \{\alpha_1 v_1 + \ldots + \alpha_n v_n \mid \alpha_i\in F,\, v_i \in E,\, n\in\mathbb N\}
\]
Сделаем важное замечание, если $E = \varnothing$ пусто, то $\langle \varnothing \rangle$ полагаем равным нулевому подпространству (подпространству состоящему только из нуля). Это полезное и удобное соглашение можно понимать так: если берется линейная комбинация с нулевым числом слагаемых, то она равна нулю.

Заметим, что $\langle E \rangle$ является наименьшим векторным подпространством содержащим $E$. Потому, для любого подпространства $U\subseteq V$ верно $\langle U \rangle  = U$.


\paragraph{Пример}

Полезно держать перед глазами следующий пример. Пусть $V = \mathbb R^3$ -- пространство, $e_1 =(1,0,0)$, $e_2 = (0,1,0)$, $e_3 = (0,0,1)$ три вектора вдоль координатных осей. Тогда $\langle e_1\rangle$, $\langle e_2\rangle$ и $\langle e_3\rangle$ -- это в точности координатные оси. Подпространства $\langle e_1, e_2\rangle$, $\langle e_1, e_3\rangle$ и $\langle e_2, e_3\rangle$ -- это плоскости содержащие пары координатных осей, $\langle e_1, e_2, e_3\rangle$ будет совпадать со всем пространством $\mathbb R^3$.


\subsection{Базис}

Подмножество $E\subseteq V$ называется порождающим, если $\langle E \rangle  = V$. Другими словами, $E$ является порождающим если любой вектор из $V$ является линейной комбинацией векторов из $E$. Отметим, что $V$ целиком всегда является порождающим. Если $E\subseteq E'\subseteq V$ и подмножество $E$ является порождающим, то и $E'$ тоже порождающее. Потому порождающее семейство всегда можно увеличить и это не интересно, интереснее попытаться его уменьшить и сделать более экономным. Порождающее множество $E$ называется минимальным, если любое строго меньшее подмножество $E$ уже не порождающее. Для этого достаточно проверить, что для любого $v\in E$ множество $E\setminus \{v\}$ уже не порождающее.

Подмножество $E\subseteq V$ называется линейно независимым, если любое конечное подмножество векторов $E$ линейно независимо. Если $E'\subseteq E\subseteq V$ и $E$ является линейно независимым, то $E'$ тоже будет линейно независимым. Потому линейно независимое подмножество можно всегда уменьшать\footnote{Вопрос линейной независимости пустого множества оставим на совести строгой аксиоматической теории множеств и не будем его касаться, чтобы не обжечься о всякий формальный геморрой.} и это не интересно, интереснее попытаться его увеличить и сделать наиболее большим. Линейно независимое подмножество $E$ называется максимальным, если любое строго содержащее его подмножество является линейно зависимым. Для этого достаточно проверить, что для любого $v\in V\setminus E$ множество $E\cup \{v\}$ является линейно зависимым.

\begin{claim}\label{claim::Basis}
Пусть $V$ -- некоторое векторное пространство над некоторым полем $F$. Тогда следующие условия на подмножество $E\subseteq V$ эквивалентны:
\begin{enumerate}
\item $E$ -- минимальное порождающее подмножество.
\item $E$ -- максимальное линейно независимое подмножество.
\item $E$ -- одновременно порождающее и линейно независимое подмножество.
\end{enumerate}
\end{claim}
\begin{proof}
Будем доказывать по схеме (1)$\Leftrightarrow$(3)$\Leftrightarrow$(2).

(1)$\Rightarrow$(3). Пусть $E$ -- минимальное порождающее, нам надо показать, что оно будет линейно независимым. Предположим противное, пусть найдется вектора $v_1,\ldots,v_n\in E$ и числа $\alpha_1,\ldots,\alpha_n\in F$, так что не все из них равны нулю, что выполнено $\alpha_1 v_1 +\ldots + \alpha_n v_n = 0$. Мы можем предположить, что $\alpha_1 \neq 0$. Тогда $v_1 = \beta_2 v_2 +\ldots + \beta_n v_n$ для некоторых $\beta_i\in F$. Давайте покажем, что тогда $E\setminus\{v_1\}$ тоже является порождающим, что будет противоречить минимальности $E$. Действительно, пусть $v\in V$ -- произвольный вектор. Так как $E$ -- порождающее, то $v$ выражается через вектора из $E$, $v = \sum_i \alpha_i' v_i'$. Если среди $v_i'$ нет вектора $v_1$ то мы выразили $v$ через $E\setminus\{v_1\}$, если есть то подставим вместо него выражение $\beta_2 v_2 + \ldots + \beta_n v_n$ и получим выражение $v$ только через вектора из $E\setminus \{v_1\}$, что и требовалось.

(3)$\Rightarrow$(1). Пусть $E$ одновременно порождающее и линейно независимое, нам надо показать, что оно минимальное порождающее. Достаточно проверить, что для любого $v\in E$ вектор $v$ не лежит в $\langle E\setminus\{v\}\rangle$. Действительно, пусть лежит, тогда найдутся вектора $v_1,\ldots, v_n\in E\setminus\{v\}$ и числа $\alpha_i\in F$ такие, что $v = \alpha_1 v_1 + \ldots + \alpha_n v_n$, то тогда $(-1)v + \alpha_1 v_1 + \ldots + \alpha_n v_n = 0$ -- нетривиальная линейная комбинация разных элементов $E$, что противоречит линейной независимости $E$.

(2)$\Rightarrow$(3). Пусть $E$ -- максимальное линейно независимое, нам надо показать, что оно будет порождающим. Нам надо показать, что $\langle E \rangle  = V$. Пусть пусть это не так, возьмем $v\in V\setminus \langle E \rangle$, тогда множество $E\cup \{v\}$ строго больше, а значит линейно зависимо. То есть для каких-то $v_1,\ldots,v_n \in E\cup \{v\}$ и чисел $\alpha_i \in F$ (так что не все из них нули) выполнено $\alpha_1 v_1 + \ldots + \alpha_n v_n = 0$. Выкинув все нулевые слагаемые, можем считать, что на самом деле все $\alpha_i$ не равны нулю. Если среди $v_i$ нет $v$, то значит все они из $E$. Тогда это означает, что $E$ линейно зависимо, что неправда. Значит один из $v_i$ -- это $v$. Будем считать, что $v_1 = v$. Так как по нашему предположению все коэффициенты не нулевые, то $v = v_1$ выражается через остальные $v_2,\ldots,v_n$. Но это означает, что $v\in \langle E\rangle$, противоречие с выбором $v$.

(3)$\Rightarrow$(2). Пусть $E$ одновременно порождающее и линейно независимое, нам надо показать, что оно максимальное линейно независимое. Для этого возьмем любой вектор $v\in V\setminus E$ и покажем, что $E \cup \{v\}$ линейно зависимо. Действительно, мы знаем, что $E$ порождающее, значит $v$ представляется в виде линейной комбинации векторов из $E$, то есть $v = \alpha_1 v_1 + \ldots + \alpha_n v_n$ для некоторых $v_i\in E$ и $\alpha_i\in F$. Но тогда $(-1)v + \alpha_1 v_1 + \ldots + \alpha_n v_n  0$ -- нетривиальная линейная комбинация векторов из $E\cup \{v\}$, то есть последнее множество линейно зависимо, что и требовалось.
\end{proof}

Пусть $V$ -- векторное пространство над некоторым полем $F$, тогда подмножество $E\subseteq V$ удовлетворяющее одному из трех эквивалентных условий предыдущего утверждения называется базисом $V$.

\paragraph{Примеры}
\begin{enumerate}
\item Пусть $V = F^n$, тогда вектора
\[
v_1 = 
\begin{pmatrix}
{1}\\{0}\\{\vdots}\\{0}\\
\end{pmatrix},
v_2 = 
\begin{pmatrix}
{0}\\{1}\\{\vdots}\\{0}\\
\end{pmatrix},
\ldots,
v_n = 
\begin{pmatrix}
{0}\\{0}\\{\vdots}\\{1}
\end{pmatrix}
\]
являются базисом. Очевидно, что эти вектора линейно независимы и любой вектор через них выражается.

\item Пусть $V = F[x]$ -- множество многочленов, тогда в качестве базиса можно взять $E = \{1,x,x^2, \ldots , x^n,\ldots\}$ -- множество всех степеней $x$. Заметим, что в данном случае базис получается бесконечным.

\item Пусть $X$ -- произвольное множество и $V = \{f\colon X\to F\}$ -- множество всех функций на $X$ со значениями в $F$. Тогда это векторное пространство над $F$ с очень любопытным свойством. 

Ситуация с базисами тут устроена так. Чтобы работать с бесконечными множествами нам нужно использовать аккуратно определенную теорию множеств. Я не буду вдаваться в подробности, что там да как строится, но важно понимать, что в теории множеств вообще говоря не всякое утверждение является доказуемым или опровергаемым. Подобные утверждения можно включить в качестве дополнительных аксиом, а можно их отрицания использовать в качестве таких же законных аксиом и будут получаться совершенно разные теории множеств. Есть такая популярная аксиома <<аксиома выбора>>, которую очень любят включать в список стандартных. 

Если вы используете аксиому выбора, то можно доказать, что всякое векторное пространство имеет базис. Если же вы не используете аксиому выбора, то нельзя ни доказать, ни опровергнуть существования базиса уже в пространстве $V$ из этого примера. Оказывается, что факт существования базиса является более слабым утверждением, чем аксиома выбора. Кроме того, если базис существует по аксиоме выбора, то это значит, что не существует никакой процедуры, которая бы помогла вам описать этот базис, потому что существование подобной процедуры дало бы вам доказательство существования базиса без аксиомы выбора. 
\end{enumerate}

\paragraph{Замечания}

\begin{itemize}
\item Пусть $V$ -- некоторое векторное пространство и $E'\subseteq V$ -- произвольное линейно независимое подмножество. Тогда его всегда можно дополнить до базиса $E \supseteq E'$, потому что базис -- это максимальное линейно независимое подмножество. В случае, если существует конечный базис, это просто. А если конечного не существует, то тут придется обращаться к аккуратной формулировке аксиоматики теории множеств.

\item Пусть $V$ -- некоторое векторное пространство и $E''\subseteq V$ -- произвольное порождающее множество, тогда из него всегда можно выбрать базис $E\subseteq E''$. Как и в предыдущем случае, если существует конечный базис, то это просто. А если нет конечного базиса, то это требует аккуратной аксиоматики теории множеств.
\end{itemize}


\ProvidesFile{lecture21.tex}[Лекция 21]

\subsection{Существование тензорного произведения}

\paragraph{Достаточное условие универсальности}

Теперь когда мы знаем, что наша вожделенная операция одна единственная на всем белом свете, мы очень хотим ее найти и пощупать руками ее свойства. Но прежде чем отправиться на поиски, давайте сформулируем полезное достаточное условие универсальности. Оно поможет нам разглядеть ту самую.

\begin{claim}\label{claim::UniversalBil}
Пусть $V$, $U$ и $T$ -- векторные пространства над полем $F$, $e_1,\ldots,e,_n$ -- базис $V$, $f_1,\ldots,f_m$ -- базис $U$ и $\mu\colon V\times U\to T$ -- билинейная операция. Если векторы $\mu(e_i,f_j)$ образуют базис $T$, где $1\leqslant i\leqslant n$ и $1\leqslant j\leqslant m$, то операция $\mu$ является универсальным умножением.
\end{claim}
\begin{proof}
Пусть $\mu'\colon V\times U \to T'$ -- любая другая операция умножения. Нам надо построить линейное отображение $\varphi\colon T\to T'$ такое, что для любых векторов $v\in V$ и $u\in U$ было выполнено $\mu'(v, u) = \varphi(\mu(v, u))$. В частности должно выполняться условие $\mu'(e_i, f_j) = \varphi(\mu(e_i, f_j))$. Так как векторы $\mu(e_i, f_j)$ являются базисом $T$, то существует единственное линейное отображение $\varphi\colon T\to T'$ такое, что $\varphi(\mu(e_i, f_j)) = \mu'(e_i, f_j)$ (утверждение~\ref{claim::LinMapExist}).

Давайте проверим, что $\varphi$ удовлетворяет необходимому требованию. Пусть $v\in V$ и $u\in U$, разложим их по базисам, получим $v = \sum_{i=1}^n a_i e_i$ и $u = \sum_{j=1}^m b_j f_j$ для некоторых $a_i, b_j\in F$. Тогда
\[
\varphi(\mu(v, u)) = \varphi(\mu(\sum_{i=1}^n a_i e_i, \sum_{j=1}^m b_j f_j)) = \sum_{i,j}a_i b_j \varphi(\mu(e_i, f_j)) =  \sum_{i,j}a_i b_j \mu'(e_i, f_j) = \mu'(\sum_{i=1}^n a_i e_i, \sum_{j=1}^m b_j f_j) = \mu'(v, u)
\]
\end{proof}

\paragraph{Конструкция тензорного произведения}

Лучший способ  найти лучшую операцию -- построить ее явно. Пусть $V$ и $U$ -- два векторных пространства над полем $F$. Мы будем предполагать, что они оба конечномерны, но конструкция будет работать и в общем случае, надо лишь аккуратно произносить все заклинания. Я предъявлю конструкцию, которая будет зависеть от базиса, но в силу единственности универсальной операции, стартуя с любого базиса, мы получим одно и то же пространство и операцию.

Пусть $e_1,\ldots,e_n$ -- базис $V$ и $f_1,\ldots,f_m$ -- базис $U$. Рассмотрим набор букв $t_{ij}$, где $1\leqslant i\leqslant n$ и $1\leqslant j \leqslant m$. Положим в качестве множество формальных линейных комбинаций букв $t_{ij}$ с коэффициентами из поля\footnote{Если вы боитесь слова <<буквы>> и сочетания <<формальная линейная комбинация>>, то можете думать про эти выражения, как про многочлены.}, то есть
\[
T = \{\sum_{i,j}a_{ij}t_{ij}\mid a_{ij}\in F\}
\]
То есть элементом $T$ является картинка вида $a_{11}t_{11} + a_{12}t_{12} + \ldots + a_{nm}t_{nm}$. Две такие картинки совпадают тогда и только тогда, когда у них все коэффициенты одинаковые. Здесь умножение коэффициента $a_{ij}$ на букву $t_{ij}$ чисто формальное, мы просто приписываем коэффициент к букве, никакого другого умножения коэффициента на букву нет.

Теперь надо на этом множестве определить операции сложения и умножения на скаляры. Для полноты картины я напишу формулы:
\begin{itemize}
\item Элементы $T$ складываются покоэффициентно, то есть
\[
\sum_{ij}a_{ij}t_{ij} + \sum_{ij}b_{ij}t_{ij} = \sum_{ij}(a_{ij}+b_{ij})t_{ij}
\]
\item Элементы $T$ умножаются на числа покоэффициентно, то есть
\[
\lambda(\sum_{ij} a_{ij}t_{ij}) = \sum_{ij} \lambda a_{ij}t_{ij}
\]
\end{itemize}
Я позволю себе сказать, что аксиомы векторного пространства проверяются методом пристального взгляда и куда полезнее их проверку оставить читателю.


Теперь надо построить операцию $\mu\colon V\times U\to T$. Пусть $v\in V$ и $u\in U$. Разложим векторы по базисам и получим $v = \sum_i x_i e_i$ и $u = \sum_j y_j f_j$. Так как координаты векторов определены однозначно, мы можем определить отображение множеств $\mu\colon V\times U\to T$ по правилу
\[
\mu(v, u) = \sum_{ij}x_iy_j t{ij}
\]
Теперь надо проверить билинейность этого отображения. Действительно, пусть $v' = \sum_{i} x_i'e_i$, тогда
\[
\mu(v + v', u) = \sum_{ij}(x_i + x_i')y_j t_{ij} = \sum_{ij}x_iy_j t_{ij} +\sum_{ij}x_i'y_j t_{ij}  =\mu(v, u) + \mu(v', u)
\]
И для любого числа $\lambda \in F$ выполнено
\[
\mu(\lambda v, u) = \sum_{ij} (\lambda x_i)y_jt_{ij} = \lambda \sum_{ij}x_iy_jt_{ij} = \lambda\mu(v, u)
\]
Аналогично проверяется линейность по второму аргументу.

\begin{claim*}
Пространство $T$ вместе с операцией $\mu\colon V\times U\to T$ построенные выше дают универсальную операцию умножения на $V$ и $U$.
\end{claim*}
\begin{proof}
Заметим, что по построению у нас получилась билинейная операция такая, что $\mu(e_i, f_j) = t_{ij}$ -- базис пространства $T$. Тогда по утверждению~\ref{claim::UniversalBil} эта операция является универсальной.
\end{proof}

\paragraph{Обозначения}
\begin{itemize}
\item Построенное пространство $T$ выше будем обозначать $V\otimes_F U$ или кратко $V\otimes U$, когда понятно о каком поле идет речь. Это пространство называется тензорным произведением пространств $V$ и $U$ над полем $F$, а его элементы называются тензорами.

\item Операция $\mu$ в операторной форме обозначается символом $\otimes$. Тогда запись для операции имеет вид $\otimes \colon V\times U \to V\otimes U$ по правилу $(v,u) \mapsto v\otimes u$.

\item Если я возьму произвольный набор векторов $v_1,\ldots,v_k\in V$ и такой же длины набор $u_1,\ldots, u_k\in U$, то я могу построить элементы $v_1\otimes u_1,\ldots, v_k\otimes u_k\in V\otimes_F U$, просто перемножим векторы с помощью универсальной операции. Кроме того, я могу сложить получившиеся векторы. Потому в тензорном произведении пространств есть элементы вида $\sum_{i=1}^k v_i \otimes u_i$. Оказывается, что любой элемент тензорного произведения имеет такой вид, это показано в следующем пункте.

\item В обозначениях выше, элементы $t_{ij}$ представляются в виде $t_{ij} = e_i \otimes f_j$. А так как $t_{ij}$ -- это базис, то любой элемент $V\otimes_F U$ единственным образом представляется в виде $\sum_{ij}a_{ij}e_i\otimes f_j$. В частности любой элемент $V\otimes_F U$ имеет вид $\sum_{i=1}^k v_i\otimes u_i$ для достаточно большого $k$.

\item Тензоры вида $v\otimes u\in V\otimes_F U$ называются разложимыми. Примеры ниже покажут, что разложимых тензоров сильно меньше, чем неразложимых.\footnote{Один из примеров ниже показывает, что разложимые тензоры в гем соответствуют матрицам ранга не более $1$.}

\end{itemize}

\paragraph{Замечание о базисах}

Мы построили тензорное произведение пространств исходя из соображения, что тензорное произведение двух фиксированных базисов является базисом тензорного произведения пространств. А что будет, если взять любые другие базисы? Оказывается, они тоже дадут базис тензорного произведения.

\begin{claim}
Пусть $V$ и $U$ -- векторные пространства над полем $F$. Тогда для любых базисов $e_1',\ldots,e_n'\in V$ и $f_1',\ldots,f_m'\in U$ следует, что набор векторов $e_i' \otimes f_j'$ будет базисом в $V\otimes_F U$.
\end{claim}
\begin{proof}
Пусть $\otimes \colon V\times U\to V\otimes_F U$ -- тензорное произведение. Построим другую его версию $\mu\colon V\times U \to T$ с помощью базисов $e_i'$ и $f_j'$ по конструкции выше. Тогда по построению $\mu(e_i', f_j')$ будет базисом $T$. Но мы знаем, что универсальная операция единственная, потому пространство $T$ и $V\otimes_F U$ изоморфны и при этом изоморфизме одна операция превращается в другую (утверждение~\ref{claim::UniversalBilUnique}). Более формально, коммутативна следующая диаграмма
\[
\xymatrix@R=10pt{
	{}&{}&{V\otimes_F U}\\
	{V\times U}\ar[urr]^{\otimes}\ar[drr]_{\mu}&{}&{}\\
	{}&{}&{T}\ar@{-->}[uu]^{\phi}\\
}
\]
где $\phi\colon T\to V\otimes_F U$ является изоморфизмом. С другой стороны из коммутативности диаграммы следует, что элементы $\mu(e_i', f_j')$ переходят в $e_i'\otimes f_j'$ под действием $\phi$. А значит последние тоже являются базисом.
\end{proof}

\begin{claim}
Для любых векторных пространств $V$ и $U$ над полем $F$ верно, что $\dim_F V\otimes_F U = \dim_F V \dim_F U$.
\end{claim}
\begin{proof}
Это непосредственно следует из конструкции тензорного произведения, которая приведена выше.
\end{proof}

\subsection{Примеры тензорных произведений}

Мы очень долго говорили про нашу ненаглядную операцию, но что-то пока ни разу на нее не взглянули и не притронулись ни к одному из ее загадочных свойств. Давайте  наконец-то перейдем к примерам.

\begin{itemize}
\item Давайте рассмотрим операцию умножения вектора на число $\mu \colon V\times F \to V$ по правилу $(v, \lambda)\mapsto \lambda v$. Утверждается, что эта операция универсальная. Действительно, для этого надо воспользоваться утверждением~\ref{claim::UniversalBil} и проверить, что она базисы отправляет в базис. Пусть $e_1,\ldots,e_n$ -- базис $V$, а в качестве базиса $F$ выберем $1$. Тогда $\mu(e_1, 1) = e_1, \ldots, \mu(e_n, 1) = e_n$ -- базис $V$. Что доказывает универсальность операции. Нарисуем соответствующую коммутативную диаграмму:
\[
\xymatrix@R=10pt{
	{}&{}&{V\otimes_F F}\ar@{-->}[dd]^{\phi}\\
	{V\times F}\ar[urr]^{\otimes}\ar[drr]_{\mu}&{}&{}\\
	{}&{}&{V}\\
}
\quad
\xymatrix@R=10pt{
	{}&{}&{v\otimes \lambda}\ar@{|-->}[dd]^{\phi}\\
	{(v,\lambda)}\ar@{|->}[urr]^{\otimes}\ar@{|->}[drr]_{\mu}&{}&{}\\
	{}&{}&{\lambda v}\\
}
\]
Таким образом тензорное произведение $V\otimes_F F = V$, при этом отождествлении тензорное произведение превращается в умножение вектора на скаляр, а разложимые тензоры $v\otimes \lambda$ переходят в $\lambda v$.

\item Теперь изучим тензорное произведение $F^n \otimes_F F^m$. Рассмотрим билинейную операцию $\mu\colon F^n\times F^m \to \operatorname{M}_{n\,m}(F) $ по правилу $(x, y)\mapsto xy^t$, то есть пару векторов переводим в матрицу ранга не более $1$. Пусть $e_1,\ldots,e_n$ -- стандартный базис $F^n$ и $f_1,\ldots,f_m$ -- стандартный базис $F^m$. Тогда $\mu(e_i, f_j) = E_{ij}$ -- матричные единицы, а они образуют базис пространства $\operatorname{M}_{n\,m}(F)$. Утверждение~\ref{claim::UniversalBil} тогда влечет, что данная операция является универсальной. Нарисуем соответствующую коммутативную диаграмму
\[
\xymatrix@R=10pt{
	{}&{}&{F^n\otimes_F F^m}\ar@{-->}[dd]^{\phi}\\
	{F^n\times F^m}\ar[urr]^{\otimes}\ar[drr]_{\mu}&{}&{}\\
	{}&{}&{\operatorname{M}_{n\,m}(F)}\\
}
\quad
\xymatrix@R=10pt{
	{}&{}&{x\otimes y}\ar@{|-->}[dd]^{\phi}\\
	{(x, y)}\ar@{|->}[urr]^{\otimes}\ar@{|->}[drr]_{\mu}&{}&{}\\
	{}&{}&{x y^t}\\
}
\]
Таким образом тензорное произведение $F^n \otimes_F F^m$ совпадает с матрицами $\operatorname{M}_{n\,m}(F)$, при этом отождествлении операция тензорного произведения превращается в операцию умножения столбца на строку $(x, y)\mapsto xy^t$, а разложимый тензор $x\otimes y$ переходит в матрицу ранга не более $1$ вида $xy^t$.

Этот пример объясняет смысл определения тензорного ранга. Один из подходов к понятию ранга тензора -- минимальное количество разложимых тензоров, в сумму которых раскладывается данный тензор. Это определение превращается в определение тензорного ранга матриц, когда мы отождествляем $F^n\otimes_F F^m$ с $\operatorname{M}_{n\,m}(F)$.

\item Теперь покажем, как на тензорном языке определить линейные отображения. Пусть $V$ и $U$ -- два векторных пространства и $\Hom_F(V, U)$ -- линейные отображения из $V$ в $U$ над полем $F$. Давайте определим билинейную операцию $\mu \colon U\times V^* \to \Hom_F(V, U)$. Чтобы определить такую операцию, нам надо для каждой пары $(u,\xi)$, где $\xi \in V^*$ и $u\in U$, задать линейное отображение $\phi_{u\,\xi}\colon V\to U$. Определим это отображение по правилу $\phi_{u\,\xi}(v) = \xi(v)u$. Это значит, что мы построили отображение $\mu\colon U\times V^* \to \Hom_F(V, U)$ по правилу $(u,\xi)\mapsto \phi_{u\,\xi}$. 

Давайте проверим, что эта операция является универсальной. Пусть $e_1,\ldots,e_n\in V$ -- некоторый базис $V$, $f_1,\ldots,f_m\in U$ -- некоторый базис $U$ и $\xi_1,\ldots,\xi_n\in V^*$ -- базис $V^*$ двойственный к $e_i$. Нам надо показать, что $\phi_{f_i\,\xi_j}$ являются базисом $\Hom_F(V, U)$. Для этого посчитаем матрицы этих отображений в паре базисов $e_1,\ldots,e_n$ и $f_1,\ldots,f_m$. Заметим, что
\[
\phi_{f_i, \xi_j}(e_1,\ldots,e_n) = (f_1,\ldots,f_m)E_{ij},\quad\text{где $E_{ij}$ -- матричная единица}
\]
Что и требовалось. Нарисуем соответствующую коммутативную диаграмму:
\[
\xymatrix@R=10pt{
	{}&{}&{U\otimes_F V^*}\ar@{-->}[dd]^{\phi}\\
	{U\times V^*}\ar[urr]^{\otimes}\ar[drr]_{\mu}&{}&{}\\
	{}&{}&{\Hom_F(V, U)}\\
}
\quad
\xymatrix@R=10pt{
	{}&{}&{u\otimes \xi}\ar@{|-->}[dd]^{\phi}\\
	{(u, \xi)}\ar@{|->}[urr]^{\otimes}\ar@{|->}[drr]_{\mu}&{}&{}\\
	{}&{}&{\phi_{u\,\xi}}\\
}
\]
Таким образом тензорное $U\otimes_F V^*$ произведение можно отождествить с линейными отображениями $\Hom_F(V, U)$, при этом отождествлении операция тензорного произведения совпадает с операцией $(u, \xi)\mapsto \phi_{u\,\xi}$, а разложимый тензор $v\otimes \xi$ соответствует отображению $\phi_{u\,\xi}$.


\end{itemize}

\subsection{Индуцированное линейное отображение}

Пусть $\phi\colon V\to U$ и $\psi\colon W\to E$ -- линейные отображения векторных пространств над полем $F$. Давайте я озвучу план  действий на сейчас. Мы хотим определить линейное отображение, которое будет обозначаться $\phi\otimes \psi\colon V\otimes_F W\to U\otimes_F E$. То есть мы перемножаем тензорно области определения двух отображений и их области значений, и на них естественным образом возникает новое отображение, обозначаемое $\phi\otimes\psi$. Элемент вида $\sum_i v_i\otimes w_i$ оно будет переводить в $\sum_i \phi(v_i) \otimes \psi(w_i)$. Мы могли бы не сильно мучиться и просто определить это отображение такой формулой, но так как запись тензора в виде такой суммы неоднозначно, то вообще говоря не понятно, почему это определение корректно. Потому придется чуть-чуть потыкаться носом в формализм, а потом, утерев нос и слезу, пойти радостно применять полученные знания.

\paragraph{Построение индуцированного отображения}

Пусть $\phi\colon V\to U$ и $\psi\colon W\to E$ -- линейные отображения векторных пространств над полем $F$. Рассмотрим следующую диаграмму
\[
\xymatrix{
	{V\times W}\ar[d]_{\phi\times \psi}\ar[r]^{\otimes}\ar[rd]^{\mu}&{V\otimes_F W}\\
	{U\times E}\ar[r]^{\otimes}&{U\otimes_F E}
}
\quad
\xymatrix{
	{(v, w)}\ar@{|->}[d]_{\phi\times\psi}\ar@{|->}[r]^{\otimes}\ar@{|->}[rd]^{\mu}&{v\otimes w}\\
	{(\phi(v), \psi(w))}\ar@{|->}[r]^{\otimes}&{\phi(v)\otimes\psi(w)}
}
\]
В ней горизонтальные стрелки -- это операции тензорного произведения  на соответствующих пространствах. Вертикальная стрелка слева -- это отображение множеств, когда мы на паре векторов действуем соответствующим линейным отображением. А диагональная стрелка $\mu$ -- это композиция вертикальной стрелки и нижнего тензорного произведения.

Методом пристального взгляда определяем, что отображение $\mu\colon V\times W\to U\otimes_F E$ является билинейной (но НЕ обязательно универсальной!) операцией на $V\times W$. Потому по универсальности $V\otimes_F W$ существует линейное отображение $V\otimes_F W\to U\otimes_F E$ такое, что диаграмма ниже коммутативна
\[
\xymatrix{
	{V\times W}\ar[d]_{\phi\times \psi}\ar[r]^{\otimes}\ar[rd]^{\mu}&{V\otimes_F W}\ar@{-->}[d]\\
	{U\times E}\ar[r]^{\otimes}&{U\otimes_F E}
}
\quad
\xymatrix{
	{(v, w)}\ar@{|->}[d]_{\phi\times\psi}\ar@{|->}[r]^{\otimes}\ar@{|->}[rd]^{\mu}&{v\otimes w}\ar@{|-->}[d]\\
	{(\phi(v), \psi(w))}\ar@{|->}[r]^{\otimes}&{\phi(v)\otimes\psi(w)}
}
\]
Полученное пунктиром отображение и обозначается $\phi\otimes \psi$. Кроме того, по построению мы видим, что $(\phi\otimes\psi)(v\otimes w) = \phi(v)\otimes\psi(w)$. А значит по линейности
\[
(\phi\otimes \psi)(\sum_i v_i \otimes w_i) = \sum_i\phi(v_i) \otimes \psi(w_i)
\]


\paragraph{Линейная независимость с векторными коэффициентами}

Теперь я хочу продемонстрировать, как можно применять индуцированные линейные отображения.

\begin{claim}
Пусть $V$ и $U$ -- векторные пространства над полем $F$, $v_1,\ldots,v_n\in V$ -- линейно независимый набор векторов, $u_1,\ldots, u_n\in U$ -- какой-то набор векторов из $U$. Тогда если $v_1\otimes u_1 + \ldots + v_n \otimes u_n = 0$ в $V\otimes_F U$, то все $u_i = 0$.
\end{claim}
\begin{proof}
Давайте для каждого $i$ построим линейное отображение $\phi_i\colon V\otimes_F U \to U$ такое, что линейная комбинация $v_1\otimes u_1 + \ldots + v_n \otimes u_n$ перейдет в $u_i$. Тогда это автоматически докажет утверждение. Действительно, тогда
\[
0 = \phi_i(0) = \phi_i (v_1\otimes u_1 + \ldots + v_n \otimes u_n) = u_i
\]
Осталось лишь построить отображения с нужными свойствами. Для начала надо заметить, что мы можем построить линейные функции $\xi_i\colon V\to F$ по правилу $\xi_i(v_j) = 1$ если $i = j$ и $0$ при $i\neq j$. Например, можно дополнить $v_1,\ldots,v_n$ до базиса и потом взять двойственный базис в $V^*$. Теперь можем рассмотреть следующую композицию
\[
\phi_i\colon V\otimes_F U \stackrel{\xi_i\otimes\Identity}{\longrightarrow} F\otimes_F U = U\quad\text{где}\quad
v\otimes u \mapsto \xi_i(v)\otimes u \mapsto \xi_i(v)u
\]
Теперь проверим, что полученные отображения обладают нужным свойством:
\[
\phi_i(v_1\otimes u_1+\ldots + v_n\otimes u_n) = \phi_i(v_1\otimes u_1) + \ldots + \phi_i(v_n \otimes u_n) = \xi_i(v_1)u_1 + \ldots + \xi_i(v_n) u_n = u_i
\]
Что и требовалось.
\end{proof}

\paragraph{Замечания}
\begin{itemize}
\item Существует и другой способ показать справедливость этого утверждения, разложив все по базису тензорного произведения. А именно, можно дополнить $v_1,\ldots,v_n$ до базиса $V$, скажем $v_1,\ldots, v_k$. Выбрать произвольный базис $f_1,\ldots,f_m$ в $U$ и разложить $u_i$ по $f_j$. После чего тензор $v_1\otimes u_1 + \ldots + v_n\otimes u_n$ раскладывается по базисе $v_i\otimes f_j$ и его коэффициенты приравниваются к нулю. Очень рекомендую проделать этот способ.

\item Неформально это утверждение означает, что если векторы $v_1,\ldots,v_n$ были линейно независимы с числовыми коэффициентами в $V$, то они остаются линейно независимы в $V\otimes_F U$ с векторными коэффициентами из $U$.
\end{itemize}

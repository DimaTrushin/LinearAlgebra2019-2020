\ProvidesFile{lecture18.tex}[Лекция 18]

\subsection{Структура векторного пространства с оператором}

Изучение структуры матрицы линейного оператора в некотором базисе равносильна изучению инвариантных подпространств пространства $V$. Теорема о жордановой нормальной форме может рассматриваться таким образом как структурная теорема для векторного пространства с оператором. В общем виде теорема о жордановой нормальной форме доказывается в два шага: все пространство раскладывается в прямую сумму корневых, после чего задача сводится к нильпотентному оператору. Первый шаг мы уже на самом деле проделали в утверждении~\ref{claim::GenRootDec}.

\begin{claim}\label{claim::RootSpaceDec}
Пусть $\varphi\colon V\to V$ -- линейный оператор такой, что его характеристический многочлен раскладывается на линейные множители. Тогда $V = V^{\lambda_1} \oplus \ldots\oplus V^{\lambda_r}$, где $\spec_F \varphi = \{\lambda_1,\ldots, \lambda_r\}$.
\end{claim}
\begin{proof}
Это частный случай утверждения~\ref{claim::GenRootDec}.
\end{proof}

\paragraph{Замечание}

Давайте объясним, что мы доказали на данный момент. Пусть $\varphi\colon V\to V$ -- некоторый линейный оператор, у которого характеристический многочлен раскладывается на линейные множители. Тогда $V = V^{\lambda_1}\oplus\ldots\oplus V^{\lambda_r}$. Выберем базис $e_i$ в каждом $V^{\lambda_i}$. Тогда по одному из определений прямой суммы $e = e_1 \sqcup \ldots \sqcup e_r$ будет базисом $V$. Так как все $V^{\lambda_i}$ инвариантны относительно $\varphi$, то когда мы запишем его матрицу в этом базисе, мы получим блочно диагональную матрицу вида
\[
A_\varphi = 
\begin{pmatrix}
{A_1}&{}&{}&{}\\
{}&{A_2}&{}&{}\\
{}&{}&{\ddots}&{}\\
{}&{}&{}&{A_r}\\
\end{pmatrix}
\]
где $A_i$ -- это матрица ограничения $\varphi|_{V^{\lambda_i}}$, то есть ее размер равен $\dim V^{\lambda_i}$. Кроме того, из утверждения~\ref{claim::OperatorUpperTriangle} следует, что в каждом $V^{\lambda_i}$ можно найти такой базис, что матрица $A_i$ будет верхне треугольной с числом $\lambda_i$ на диагонали, то есть
\[
A_i = 
\begin{pmatrix}
{\lambda_i}&{*}&{\ldots}&{*}\\
{}&{\lambda_i}&{\ldots}&{*}\\
{}&{}&{\ddots}&{\vdots}\\
{}&{}&{}&{\lambda_i}\\
\end{pmatrix}
\]
Наша цель еще улучшить вид матриц $A_i$. Оказывается, почти все элементы верхнего блока можно сделать нулевыми. Что это в точности означает и как доказывается, вы узнаете в теореме о жордановой нормальной форме.


\subsection{Отношение равенства по модулю подпространства}

\begin{definition}
Пусть $V$ -- векторное пространство, а $U\subseteq V$ -- подпространство. Тогда будем говорить, что векторы $v,w\in V$ равны по модулю $U$ и писать $v = w \pmod U$, если $v - w \in U$.\footnote{На самом деле равенство по модулю подпространства сводится к равенству в некотором новом пространстве, которое называется фактор пространством. Так как мы пока не знаем, что это такое, будем пользоваться лишь отношением равенства по модулю. Думать про него надо так же, как и про остатки в целых числах.}
\end{definition}

Например, если $V = \mathbb R^2$ -- плоскость и $U = \langle e_1\rangle$ -- горизонтальная прямая, то все векторы лежащие на горизонтальных прямых между собой равны по модулю $U$. Например, $e_2 = e_2 + e_1 = e_2 - 3 e_1 \pmod U$. Но $e_2 \neq 2 e_2 \pmod U$.

Заметим, что обычное равенство -- это равенство по модулю нулевого подпространства. С другой стороны, по модулю подпространства $U = V$ любые два вектора равны.

\begin{definition}
Пусть $V$ -- векторное пространство,  $U\subseteq V$ -- некоторое подпространство, и $v_1,\ldots,v_n\in V$ -- набор векторов.
\begin{enumerate}
\item Будем говорить, что $v_1,\ldots,v_n$ линейно независимы по модулю $U$, если из равенства $\alpha_1 v_1 + \ldots + \alpha_n v_n = 0 \pmod U$ следует, что $\alpha_1 = \ldots = \alpha_ n = 0$.

\item Будем говорить, что $v_1, \ldots, v_n$ порождающие по модулю $U$, если для любого вектора $v\in V$ найдутся коэффициенты $\alpha_1,\ldots,\alpha_n\in F$ такие, что $v = \alpha_1 v_1 + \ldots + \alpha_n v_n\pmod U$.

\item Будем говорить, что $v_1,\ldots,v_n$ являются базисом $V$ по модулю $U$, если они одновременно линейно независимы и порождающие по модулю $U$.\footnote{Так как равенство по модулю сводится к равенству в некотором новом пространстве, то все факты про базис аналогичные обычным фактам, что мы доказывали будут верны. Вам же я предлагаю доказать их по аналогии в качестве упражнения.}
\end{enumerate}
\end{definition}

\begin{claim}
Пусть $V$ -- векторное пространство, $U\subseteq V$ -- подпространство, и $v_1,\ldots,v_n\in V$ -- набор векторов. Тогда
\begin{enumerate}
\item Векторы $v_1,\ldots,v_n$ линейно независимы по модулю $U$ тогда и только тогда, когда они линейно независимы и $\langle v_1,\ldots,v_n \rangle \cap U = 0$.

\item Векторы $v_1, \ldots,v_n$ порождающие по модулю $U$ тогда и только тогда, когда $\langle v_1,\ldots,v_n\rangle + U = V$.

\item Векторы $v_1, \ldots,v_n$ являются базисом по модулю $U$ тогда и только тогда, когда они линейно независимы и $\langle v_1,\ldots,v_n\rangle \oplus U = V$.
\end{enumerate}
\end{claim}
\begin{proof}
(1) $\Rightarrow$ Пусть $\alpha_1v_1+\ldots+\alpha_n v_n = 0$, тогда $\alpha_1 v_1 + \ldots+ \alpha_n v_n \in U$. Последнее означает, что $\alpha_1v_1 + \ldots+ \alpha_n v_n = 0 \pmod U$. А значит все $\alpha_i = 0$. Значит $v_i$ линейно независимы. Теперь рассмотрим вектор $v\in \langle v_1,\ldots,v_n\rangle \cap U$. Так как $v$ лежит в первом подпространстве, то $v = \alpha_1 v_1 + \ldots+\alpha_n v_n$. Так как он лежит в правом подпространстве, то $\alpha_1 v_1 + \ldots+ \alpha_n v_n = v \in U$. Значит $\alpha_1 v_1 + \ldots + \alpha_n v_n = 0\pmod U$. А следовательно все $\alpha_i =0$. Но значит и $v = 0$, что и требовалось.

$\Leftarrow$ Пусть $\alpha_1 v_1 +\ldots+\alpha_n v_n = 0 \pmod U$. Это значит, что $\alpha_1 v_1 + \ldots + \alpha_n v_n \in U$. А значит $\alpha_1 v_1 +\ldots + \alpha_n v_n \in \langle v_1,\ldots,v_n\rangle \cap U = 0$. То есть $\alpha_1 v_1 + \ldots + \alpha_n v_n = 0$. Но так как $v_i$ линейно независимы, то $\alpha_i = 0$ для всех $i$.

(2) $\Rightarrow$ По определению, для любого $v\in V$ найдутся коэффициенты $\alpha_i$ такие, что $v = \alpha_1 v_1 + \ldots + \alpha_n v_n \pmod U$. То есть $v - (\alpha_1 v_1 + \ldots + \alpha_n v_n) = u \in U$. Значит, $v = \alpha_1 v_1 + \ldots + \alpha_n v_n + u$, что и требовалось.

$\Leftarrow$ Пусть $v\in V = \langle v_1, \ldots, v_n \rangle + U$. Тогда $v = \alpha_1 v_1 + \ldots + \alpha_n v_n + u$. По определению это означает, что $v = \alpha_1 v_1 + \ldots + \alpha_n v_n \pmod U$.

(3) Этот пункт получается из первых двух вместе взятых.
\end{proof}

\subsection{Жорданова нормальная форма для нильпотентных операторов}

\begin{definition}
Пусть $\varphi \colon V\to V$ -- нильпотентный оператор на векторном пространстве $V$ над полем $F$ и пусть $v\in V$ -- некоторый вектор. Тогда высота вектора $v$ относительно $\varphi$ это 
\[
\height_\varphi v = \min\{k\in \mathbb Z_+ \mid \varphi^k v = 0\}
\]
\end{definition}

Рассмотрим цепочку подпространств
\[
0\subset \ker \varphi \subset \ker \varphi^2 \subset \ldots \subset \ker \varphi^{k-1} \subset \ker \varphi^k = \ker \varphi^{k+1} = \ldots
\]
Нулевой вектор -- это единственный вектор высоты $0$. Векторы в $\ker \varphi$ -- это векторы высоты не больше одного, причем, векторы из $\ker \varphi \setminus 0$ -- это в точности векторы высоты $1$. Векторы из $\ker\varphi^2 \setminus \ker \varphi$ -- это векторы высоты $2$ и так далее. Так как стабилизация наступает на шаге $k$, то самая большая высота у векторов будет $k$.

\begin{definition}\label{def::JNF}
Матрица 
\[
J_n(\lambda) = 
\begin{pmatrix}
{\lambda}&{1}&{}&{}&{}\\
{}&{\lambda}&{1}&{}&{}\\
{}&{}&{\ddots}&{\ddots}&{}\\
{}&{}&{}&{\lambda}&{1}\\
{}&{}&{}&{}&{\lambda}\\
\end{pmatrix}
\in \operatorname{M}_n(F)
\]
называется жордановой клеткой размера $n$. Если матрица $A$ блочно диагональная, где на диагонали стоят жордановы клетки
\[
A = 
\begin{pmatrix}
{J_{n_1}(\lambda_1)}&{}&{}&{}\\
{}&{J_{n_2}(\lambda_2)}&{}&{}\\
{}&{}&{\ddots}&{}\\
{}&{}&{}&{J_{n_r}(\lambda_r)}\\
\end{pmatrix}
\in \operatorname{M}_{n_1 + \ldots + n_r}(F)
\]
то говорят, что $A$ имеет жорданову нормальную форму.
\end{definition}

Рассмотрим оператор $\phi\colon F^n \to F^n$, по правилу $x\mapsto J_n(0)x$. Если $e_1, \ldots ,e_n$ -- стандартный базис в $F^n$, то мы видим, что 
\[
0\stackrel{\phi}{\longleftarrow}e_1\stackrel{\phi}{\longleftarrow}e_2 \stackrel{\phi}{\longleftarrow}\ldots\stackrel{\phi}{\longleftarrow}e_{n-1}\stackrel{\phi}{\longleftarrow}e_n
\]
Наоборот, пусть у нас нашелся базис для оператора $\phi\colon V\to V$ с таким свойством. Тогда метод пристального взгляда нам подсказывает, что его матрица в этом базисе будет $J_n(0)$. 

Таким образом мы описали на геометрическом языке как понять, что в некотором базисе матрица оператора задана жордановой клеткой с нулем на диагонали.

\begin{claim}\label{claim::LinIndepModKer}
Пусть $\varphi\colon V\to V$ -- нильпотентный оператор и $v_1,\ldots,v_r\in \ker \varphi^{k+1}$ линейно независимы по модулю $\ker \varphi^k$. Тогда векторы $\varphi v_1,\ldots,\varphi v_r$ лежат в $\ker \varphi^k$ и при $k \geqslant 1$\footnote{При $k = 0$ все векторы $\varphi v_i = 0$. Формально доказательство не сработает, так как при $k = 0$ не определено $\ker \varphi^{0 - 1}$.} они линейно независимы по модулю $\ker \varphi^{k-1}$.
\end{claim}
\begin{proof}
Тот факт что $\varphi v_i\in \ker \varphi^{k}$ следует из того, что они зануляются оператором $\varphi^k$. Теперь покажем, что они линейно независимы по модулю $\ker \varphi^{k-1}$. Пусть 
\[
\alpha_1 \varphi v_1 + \ldots + \alpha_r \varphi v_r = 0 \pmod{\ker \varphi^{k-1}}
\]
Тогда
\begin{gather*}
\varphi(\alpha_1 v_1 + \ldots + \alpha_r v_r)\in \ker \varphi^{k-1}\Rightarrow\\
\varphi^k(\alpha_1 v_1 + \ldots + \alpha_r v_r) = 0\Rightarrow\\
\alpha_1v_1 +\ldots + \alpha_r v_r \in \ker\varphi^k\Rightarrow\\
\alpha_1 v_1 + \ldots + \alpha_r v_r = 0 \pmod{\ker \varphi^k}
\end{gather*}
Значит все $\alpha_i = 0$, что и требовалось.
\end{proof}

\begin{claim}[ЖНФ для нильпотентов]\label{claim::NilJNF}
Пусть $\varphi\colon V\to V$ -- нильпотентный оператор. Тогда
\begin{enumerate}
\item Для оператора $\varphi$ существует жорданов базис, то есть в некотором базисе матрица $\varphi$ имеет вид
\[
A_\varphi = 
\begin{pmatrix}
{J_{n_1}(0)}&{}&{}&{}\\
{}&{J_{n_2}(0)}&{}&{}\\
{}&{}&{}&{}\\
{}&{}&{}&{J_{n_r}(0)}\\
\end{pmatrix}
\]
\item Количество клеток размера $r$ вычисляется по формуле
\[
2\dim \ker \varphi^r - \dim \ker \varphi^{r+1} - \dim \ker \varphi^{r-1}
\]
Так как эти числа не зависят от базиса, то в любом жордановом базисе количество клеток размера $r$ одинаковое. А значит жордановы формы в разных базисах могут отличаться лишь перестановкой клеток.
\end{enumerate}
\end{claim}
\begin{proof}
(1) Пусть $\varphi^k = 0$, причем $k$ -- наименьшее возможное. Для того, чтобы доказать теорему, мне надо найти базис в пространстве $V = \ker \varphi^k$, состоящий из цепочек вида:
\[
0\stackrel{\varphi}{\longleftarrow}e_1\stackrel{\varphi}{\longleftarrow}e_2 \stackrel{\varphi}{\longleftarrow}\ldots\stackrel{\varphi}{\longleftarrow}e_{n-1}\stackrel{\varphi}{\longleftarrow}e_n
\]
Каждая такая цепочка будет давать одну клетку размера $n$. Так как вектор $e_n$ в такой цепочке имеет высоту $n$ его надо искать в $\ker \varphi^{n}\setminus\ker\varphi^{n-1}$. Значит, чтобы получить самые длинные цепочки я должен как-то выбрать векторы $v_1,\ldots, v_r$ в $\ker\varphi^k \setminus \ker \varphi^{k-1}$. При этом, я хочу, чтобы все векторы вида $\varphi^i v_j$ были между собой линейно независимы. То есть выбирать надо аккуратно. Для этого мне и понадобится понятие линейной независимости по модулю подпространства. Итак, приступим.

Возьмем $v_1,\ldots,v_{r_1}\in\ker\varphi^k$ -- базис $\ker \varphi^k$ по модулю $\ker \varphi^{k-1}$. Последнее означает, что $\ker \varphi^k = \langle v_1,\ldots,v_{r_1}\rangle \oplus \ker \varphi^{k-1}$. Из утверждения~\ref{claim::LinIndepModKer} следует, что векторы $\varphi v_1,\ldots,\varphi v_{r_1}$ лежат в $\ker \varphi^{k-1}$ и линейно независимы по модулю $\ker \varphi^{k-2}$.\footnote{Либо, что $k = 1$, то есть $V = \ker \varphi$, а значит $\varphi = 0$ и доказывать нечего.} Значит их можно дополнить до базиса пространства $\ker \varphi^{k-1}$ по модулю подпространства $\varphi^{k-2}$ векторами $v_{r_1+1},\ldots,v_{r_2}$. Данный процесс можно изобразить на следующей диаграмме:
\[
\xymatrix@R=15pt@C=15pt{
  {\ker \varphi^k}&
  {v_1}\ar@{|->}[d]&{\ldots}&{v_{r_1}}\ar@{|->}[d]&
  {}&{}&{}\\
  {\ker \varphi^{k-1}}\ar@{}[u]|{\cup}&
  {\varphi v_1}&{\ldots}&{\varphi v_{r_1}}&
  {v_{r_1 + 1}}&{\ldots}&{v_{r_2}}\\
}
\]
Кроме того, проделанное означает, что 
\begin{gather*}
\ker \varphi^k = \langle v_1,\ldots, v_{r_1}\rangle \oplus \ker \varphi^{k-1}\\
\ker \varphi^{k-1} = \langle \varphi v_1,\ldots, \varphi v_{r_1}\rangle \oplus \langle v_{r_1 + 1},\ldots,v_{r_2}\rangle \oplus \ker \varphi^{k-2}
\end{gather*}
Мы можем продолжать этот процесс далее. Он остановится, когда мы дойдем до $\ker \varphi$, так как следующее подпространство будет уже нулевым. Весь процесс можно изобразить на следующей диаграмме (здесь я векторы обозначил точками, чтобы не загромождать обозначения):
\[
\xymatrix@R=15pt@C=15pt{
  {\ker\varphi^k}&{\bullet}\ar[d]
  {\save
   [].[rrr]*[F-:<3pt>]\frm{}
  \restore}
  {\save
   [].[rrr]*+[F--]\frm{}
  \restore}&
  {\bullet}\ar[d]&{\ldots}&{\bullet}\ar[d]&
  {}&{}&{}&
  {}&
  {}&{}&{}&
  {}&{}&{}\\
  {\ker\varphi^{k-1}}\ar@{}[u]|{\cup}&{\bullet}\ar[d]
  {\save
   [].[rrrrrr]*+[F--]\frm{}
  \restore}&
  {\bullet}\ar[d]&{\ldots}&{\bullet}\ar[d]&
  {\bullet}\ar[d]
  {\save
   [].[rr]*[F-:<3pt>]\frm{}
  \restore}&{\ldots}&{\bullet}\ar[d]&
  {}&
  {}&{}&{}&
  {}&{}&{}\\
  {\vdots}\ar@{}[u]|{\cup}&
  {\vdots}\ar[d]&{\vdots}\ar[d]&{\vdots}&{\vdots}\ar[d]&
  {\vdots}\ar[d]&{\vdots}&{\vdots}\ar[d]&
  {\ddots}&
  {}&{}&{}&
  {}&{}&{}\\
  {\ker\varphi^2}\ar@{}[u]|{\cup}&{\bullet}\ar[d]
  {\save
   [].[rrrrrrrrrr]*+[F--]\frm{}
  \restore}&
  {\bullet}\ar[d]&{\ldots}&{\bullet}\ar[d]&
  {\bullet}\ar[d]&{\ldots}&{\bullet}\ar[d]&
  {\ldots}&
  {\bullet}\ar[d]
  {\save
   [].[rr]*[F-:<3pt>]\frm{}
  \restore}&{\ldots}&{\bullet}\ar[d]&
  {}&{}&{}\\
  {\ker\varphi}\ar@{}[u]|{\cup}&{\bullet}\ar[d]
  {\save
   [].[rrrrrrrrrrrrr]*+[F--]\frm{}
  \restore}&
  {\bullet}\ar[d]&{\ldots}&{\bullet}\ar[d]&
  {\bullet}\ar[d]&{\ldots}&{\bullet}\ar[d]&
  {\ldots}&
  {\bullet}\ar[d]&{\ldots}&{\bullet}\ar[d]&
  {\bullet}\ar[d]
  {\save
   [].[rr]*[F-:<3pt>]\frm{}
  \restore}&{\ldots}&{\bullet}\ar[d]\\
  {0}\ar@{}[u]|{\cup}&
  {0}&{0}&{\ldots}&{0}&
  {0}&{\ldots}&{0}&
  {\ldots}&
  {0}&{\ldots}&{0}&
  {0}&{\ldots}&{0}\\
}
\]
Кроме того, мы будем иметь равенства вида:
\[
\ker \varphi^{k-s} = \langle \varphi^s v_1,\ldots,\varphi^s v_{r_1}\rangle \oplus \langle \varphi^{s-1}v_{r_1 + 1},\ldots,\varphi^{s-1}v_{r_2}\rangle \oplus \ldots \oplus \langle v_{r_s + 1},\ldots,v_{r_{s+1}} \rangle \oplus \ker \varphi^{k - s -1}
\]
То есть все векторы расположенные в заштрихованных прямоугольниках на диаграмме выше являются линейно независимыми между собой и со всеми векторами, которые лежат ниже них. Значит все построенные вектора (точки на диаграмме выше) являются базисом пространства $\ker \varphi^k = V$. А это то, что и надо было сделать.

(2) Теперь нам надо посчитать количество клеток размера $r$ в жордановой форме. Это соответствует тому, чтобы посчитать количество цепочек длины $r$ на большой диаграмме выше. То есть нам надо посчитать количество векторов обведенных в овальную рамку в строке для $\ker \varphi^r$. В начале посчитаем количество векторов в заштрихованной рамке на каждом этаже. Так как $\ker \varphi^r$ порожден всеми векторами на слое $r$ и ниже, то в строке для $\ker \varphi^r$ количество векторов в заштрихованной рамке равно $\dim \ker \varphi^r - \dim \ker \varphi^{r-1}$. Тогда количество векторов в овальной рамке на этаже $r$ равно количество векторов в заштрихованной рамке на этаже $r$ минус количество векторов в заштрихованной рамке на этаже $r+1$. Значит, искомое количество клеток размера $r$ равно:
\[
(\dim \ker \varphi^r - \dim \ker \varphi^{r-1}) - (\dim \ker \varphi^{r+1} - \dim \ker \varphi^r) = 2 \dim \ker \varphi^r - \dim \ker \varphi^{r+1} - \dim \ker \varphi^{r-1}
\]
\end{proof}

\paragraph{Замечания}
Отметим специальный вид для количества клеток максимального и минимального размеров и сделаем еще пару замечаний.
\begin{itemize}
\item Максимальный размер $r = k$. Тогда как мы видим, количество клеток равно
\[
\dim \ker \varphi^k - \dim \ker \varphi^{k-1} = 
\dim V - \dim \ker \varphi^{k-1} = \dim \Im\varphi^{k-1}
\]
То есть ранг последней ненулевой степени оператора $\varphi$ -- это количество клеток максимальной размерности.

\item Минимальный размер $r = 1$. Тогда $\ker \varphi^{r - 1} = 0$. Значит, количество клеток размера $1$, то есть, количество отдельно стоящих нулей в жордановой форме будет
\[
2 \dim \ker \varphi - \dim \ker \varphi^{2} 
\]
Обратите внимание на то, что это НЕ размерность ядра.

\item Размерность ядра $\dim \ker \varphi$ -- это количество всех клеток всевозможных размеров.

\item Максимальный размер клетки -- это степень минимального многочлена для $\varphi$ или что то же самое -- кратность его единственного корня $0$.
\end{itemize}

\subsection{Теорема о жордановой нормальной форме}

\begin{claim}[Теорема о жордановой нормальной форме]\label{claim::JNF}
Пусть $\varphi\colon V\to V$ -- линейный оператор такой, что его характеристический многочлен раскладывается на линейные множители
\[
\chi_\varphi(t) = (t - \lambda_1)^{n_1} \ldots (t - \lambda_r)^{n_r}
\]
Тогда
\begin{enumerate}
\item Для оператора $\varphi$ существует жорданов базис, то есть в некотором базисе матрица $\varphi$ имеет вид
\[
A_\varphi = 
\begin{pmatrix}
{J_{k_1}(\lambda_{i_1})}&{}&{}&{}\\
{}&{J_{k_2}(\lambda_{i_2})}&{}&{}\\
{}&{}&{}&{}\\
{}&{}&{}&{J_{k_s}(\lambda_{i_s})}\\
\end{pmatrix}
\]
\item В любом жордановом базисе количество клеток размера $m$ с фиксированным числом $\lambda$ на диагонали одинаковое и равно
\[
2\dim \ker (\varphi - \lambda\Identity)^m - \dim \ker (\varphi - \lambda\Identity)^{m+1} - \dim \ker (\varphi - \lambda\Identity)^{m-1}
\]
А значит жордановы формы в разных базисах могут отличаться лишь перестановкой клеток.
\end{enumerate}
\end{claim}
\begin{proof}
(1) Так как $\chi_\varphi(t)$ раскладывается на линейные множители, утверждение~\ref{claim::RootSpaceDec} говорит, что $V = V^{\lambda_1}\oplus \ldots\oplus V^{\lambda_r}$. Тогда, если мы выберем базисы в подпространствах $V^{\lambda_i}$ объединим (они обязательно дадут базис $V$) и запишем в этом базисе матрицу $\varphi$, она будет иметь блочно диагональный вид
\[
A_\varphi = 
\begin{pmatrix}
{A_1}&{}&{}&{}\\
{}&{A_2}&{}&{}\\
{}&{}&{\ddots}&{}\\
{}&{}&{}&{A_r}\\
\end{pmatrix}
\]
где $A_i$ -- матрица $\varphi|_{V^{\lambda_i}}$. То есть, чтобы доказать теорему, нам надо в каждом $V^{\lambda_i}$ выбрать жорданов базис для оператора $\varphi|_{V^{\lambda_i}}$. Теперь заметим, что базис является жордановым для некоторого оператора $\phi$ тогда и только тогда, когда он является жордановым для оператора $\phi - \lambda \Identity$ (при любом выборе $\lambda$). Потому нам надо в каждом $V^{\lambda_i}$ выбрать жорданов базис для оператора $\phi_i := \varphi|_{V^{\lambda_i}} - \lambda_i \Identity$. Но оператор $\phi_i$ является нильпотентным и для него это следует из утверждения~\ref{claim::NilJNF}.

(2) Пусть теперь у нас выбран какой-нибудь жорданов базис, в котором матрица $\varphi$ имеет вид
\[
A_\varphi = 
\begin{pmatrix}
{A_1}&{}&{}&{}\\
{}&{A_2}&{}&{}\\
{}&{}&{\ddots}&{}\\
{}&{}&{}&{A_r}\\
\end{pmatrix},
\quad\text{где}\quad
A_i = 
\begin{pmatrix}
{J_{k_{1\,i}}(\lambda_i)}&{}&{}&{}\\
{}&{J_{k_{2\,i}}(\lambda_i)}&{}&{}\\
{}&{}&{}&{}\\
{}&{}&{}&{J_{k_{m_i\,i}}(\lambda_i)}\\
\end{pmatrix}
\]
Во-первых, числа $\lambda_i$ на диагоналях клеток будут обязательно числами из спектра, просто потому что $A_\varphi$ верхне треугольная с этими числами на диагонали.

Во-вторых, нам надо показать, что все $A_i$ (где $A_i$ -- это блоки в которых мы сгруппировали клетки с одним и тем же числом $\lambda_i$ на диагонали) имеют одинаковый размер. Но по определению размер этих блоков -- это кратность $\lambda_i$ в $\chi_\varphi(t)$ или что то же самое -- размерность $V^{\lambda_i}$. А сам блок $A_i$ оказывается матрицей оператора $\varphi|_{V^{\lambda_i}}$.

В-третьих, надо показать, что внутри каждого $A_i$ количество блоков фиксированного размера одинаковое  и задано формулой 
\[
2\dim \ker (\varphi - \lambda_i\Identity)^m - \dim \ker (\varphi - \lambda_i\Identity)^{m+1} - \dim \ker (\varphi - \lambda_i\Identity)^{m-1}
\]
Но так как $A_i$ -- это матрица оператора $\varphi|_{V^{\lambda_i}}$, а оператор $\varphi|_{V^{\lambda_i}} - \lambda_i \Identity$ нильпотентен и имеет те же размеры блоков, то из пункта~(2) утверждения~\ref{claim::NilJNF} следует, что нужное количество клеток задано формулой
\[
2\dim \ker (\varphi|_{V^{\lambda_i}} - \lambda_i\Identity)^m - \dim \ker (\varphi|_{V^{\lambda_i}} - \lambda_i\Identity)^{m+1} - \dim \ker (\varphi|_{V^{\lambda_i}} - \lambda_i\Identity)^{m-1}
\]
Теперь осталось показать, что 
\[
\ker (\varphi - \lambda_i\Identity)^m = \ker (\varphi|_{V^{\lambda_i}} - \lambda_i\Identity)^m
\]
Если вспомнить определение оператора ограничения мы видим, что
\[
\ker (\varphi|_{V^{\lambda_i}} - \lambda_i\Identity)^m = V^{\lambda_i}\cap \ker (\varphi - \lambda_i\Identity)^m
\]
С другой стороны, так как $\ker (\varphi - \lambda_i\Identity)^m$ является инвариантным подпространством, то из утверждения~\ref{claim::GenRootDec} пункт~(3) следует, что
\[
\ker (\varphi - \lambda_i\Identity)^m = \oplus_{j=1}^r \ker (\varphi - \lambda_i\Identity)^m \cap V^{\lambda_j}
\]
Но при $i\neq j$ оператор $\varphi - \lambda_i \Identity$ обратим на $V^{\lambda_j}$ по утверждению~\ref{claim::CoprimeKernels} пункт~(2). Значит пересечение будет не ноль только при $i = j$. А это и дает нужное равенство.
\end{proof}



Обратите внимание, что жорданова форма для оператора единственная с точностью до перестановки блоков. Однако, жордановых базисов может быть много! Например, если $\varphi = \lambda \Identity$, то любой базис является жордановым, так как в любом базисе матрица оператора будет диагональной с числом $\lambda$ на диагонали.


\paragraph{Замечания}

Давайте обсудим некоторые характеристики жордановых клеток в терминах исходного оператора.
\begin{itemize}
\item Число $\dim V^{\lambda_i}$ является суммарным размером всех клеток с заданным $\lambda_i$, то есть клеток вида $J_s(\lambda_i)$. Действительно, по построению, блок из таких клеток возникает как матрица $\varphi|_{V^{\lambda_i}}$. А размер матрицы оператора равен размерности пространства, на котором определен оператор.\footnote{Можно объяснить по-другому из явного вычисления с матрицей жордановой нормальной формы аналогично следующему пункту.}

\item Число $\dim V_{\lambda_i}$ является количеством жордановых клеток с фиксированным $\lambda_i$, то есть клеток вида $J_s(\lambda_i)$. Действительно, по определению $V_{\lambda_i} = \ker (\varphi - \lambda_i \Identity)$. Теперь надо посчитать правую часть равенства в жордановой форме и увидеть, что его размерность равна суммарному размеру клеток с числом $\lambda_i$. Пусть $A$ имеет жорданову форму как в предыдущем утверждении, тогда
\[
A -\lambda_i E= 
\begin{pmatrix}
{A_1-\lambda_i E}&{}&{}\\
{}&{\ddots}&{}\\
{}&{}&{A_r - \lambda_i E}
\end{pmatrix},\text{ где }
A_k -\lambda_i E=
\begin{pmatrix}
{J_*(\lambda_k - \lambda_i)}&{}&{}\\
{}&{\ddots}&{}\\
{}&{}&{J_*(\lambda_k - \lambda_i)}
\end{pmatrix}
\]
То есть для $k\neq i$ все блоки будут верхне треугольными с ненулевым числом на диагонали, а значит обратимыми, а для $k = i$ блок будет вида
\[
A_i -\lambda_i E=
\begin{pmatrix}
{J_*(0)}&{}&{}\\
{}&{\ddots}&{}\\
{}&{}&{J_*(0)}
\end{pmatrix}
\]
Пусть $A$ размера $n$, а $n_k$ -- размер блока $A_k$. Тогда размерность $\ker (A - \lambda_i E)$ равна $n - \rk (A - \lambda_i E)$. Но $\rk (A - \lambda_i E) = \sum_k \rk (A_k - \lambda_i E)$ и $n = \sum_k n_k$. Так как все матрицы $A_k - \lambda_i E$ для $k\neq i$ обратимы, то есть имеют полный ранг, то $n - \rk (A-\lambda_i E)$ равно $n_i - \rk (A_i - \lambda_iE)$. Каждый блок $J_s(0)$ имеет ранг $s-1$. То есть каждая клетка вносит в ранг вклад на единицу меньше размера. Следовательно $\rk(A_i -\lambda_i E)$ равно $n$ минус количество клеток, победа!

\item Кратность $\lambda_i$ в $f_\text{min}$ равна размерности самого большого блока вида $J_s(\lambda_i)$. Давайте посчитаем кратность корня $\lambda_i$ в $f_\text{min}$ для $A$ в жордановой форме. Для этого надо найти $m$ для которого наступит $\ker (A - \lambda_i E)^m = \ker(A-\lambda_i E)^{m+1}$ (утверждение~\ref{claim::RootMultGeom}). Как и выше, блоки $A_k - \lambda_i E$ не дают вклад в ядро. А блок $A_i - \lambda_i E$ зануляется в степени равной максимальному размеру клетки. То есть это ровна степень стабилизации, что и требовалось.

\item Мы уже доказали, что количество клеток размера $r$ с числом $\lambda$ вычисляется по формуле
\[
2\dim \ker (\varphi - \lambda\Identity)^r - \dim \ker (\varphi - \lambda\Identity)^{r+1} - \dim \ker (\varphi - \lambda\Identity)^{r-1}
\]
Однако, можно дать совершенно другое доказательство этого факта, которое может оказаться для вас более приятным или более понятным. Вместо всех этих дурацких рассуждений с операторами, которые были приведены во второй части предыдущего утверждения, можно поступить вот как. Пусть мы уже привели матрицу в ЖНФ в каком-то базисе (но пока еще не знаем ее единственности).
\[
A_\varphi = 
\begin{pmatrix}
{A_1}&{}&{}&{}\\
{}&{A_2}&{}&{}\\
{}&{}&{\ddots}&{}\\
{}&{}&{}&{A_r}\\
\end{pmatrix},
\quad\text{где}\quad
A_i = 
\begin{pmatrix}
{J_{k_{1\,i}}(\lambda_i)}&{}&{}&{}\\
{}&{J_{k_{2\,i}}(\lambda_i)}&{}&{}\\
{}&{}&{}&{}\\
{}&{}&{}&{J_{k_{m_i\,i}}(\lambda_i)}\\
\end{pmatrix}
\]
А давайте для данной ЖНФ просто посчитаем число 
\[
2\dim \ker (\varphi - \lambda\Identity)^r - \dim \ker (\varphi - \lambda\Identity)^{r+1} - \dim \ker (\varphi - \lambda\Identity)^{r-1}
\]
То есть мы будем считать
\[
2\dim \ker (A_\varphi - \lambda E)^r - \dim \ker (A_\varphi - \lambda E)^{r+1} - \dim \ker (A_\varphi - \lambda E)^{r-1}
\]
Проделаем это аналогично тому, как в одном из замечаний выше и увидим, что это число дает количество клеток размера $r$ для данной ЖНФ.\footnote{Попробуйте довести это рассуждение до конца, это очень полезно и просто.} Но с другой стороны, это число не зависит от ЖНФ, значит для любой ЖНФ число клеток считается по этой формуле.
\end{itemize}


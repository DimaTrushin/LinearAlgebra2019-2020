\ProvidesFile{lecture20.tex}[Лекция 20]



\subsection{Функции на функциях}

Давайте рассмотрим следующую последовательность векторных пространств
\begin{center}
\begin{tabular}{c|c|c}

{$V$}&{$V^*$}&{$V^{**}$}\\

\hline

{$v$}&{$\xi$}&{$?$}\\

\end{tabular}
\end{center}
В $V$ у нас живут векторы, в $V^*$ функции на векторах, а в $V^{**}$ -- функции на функциях на векторах. Оказывается, каждый вектор можно рассматривать как функцию на функциях на векторах. Давайте вспомним наш удобный формализм: если $v\in V$ и $\xi \in V^*$, то $\xi(v)$ надо обозначать так $\xi v$, то есть как произведение. Тогда зафиксировав $\xi$ мы получим правило линейное по $v$, то есть отображение $V\to F$. Но, мы с таким же успехом можем зафиксировать $v$ и начать менять левый аргумент. Тогда получится линейное отображение $V^* \to F$. Это ломает мозг в записи $\xi(v)$, но когда и функционал и вектор записаны равноправно в виде умножения $\xi v$, такие конструкции становится проще понимать. Теперь аккуратно.

Для произвольного векторного пространства $V$ построим отображение $\phi\colon V\to V^{**}$, где $v\mapsto \phi_v$. Чтобы задать это отображение нам надо определить элемент $\phi_v\in V^{**}$, то есть нам надо задать $\phi_v\colon V^*\to F$. То есть нам надо определить число из поля $\phi_v(\xi)$ для каждого $\xi\in V^*$. Для этого положим по определению $\phi_v(\xi) = \xi(v)$ -- функционал вычисления на векторе. Обратим внимание, что отображение $\phi\colon V\to V^{**}$ является линейным отображением между векторными пространствами.


\begin{claim}\label{claim::DoubleDuoIsom}
Пусть $V$ -- произвольное векторное пространство и $\phi\colon V\to V^{**}$ отображение по правилу $v\mapsto \phi_v$, где $\phi_v(\xi) = \xi(v)$. Тогда 
\begin{enumerate}
\item Отображение $\phi\colon V\to V^{**}$ инъективно.
\item Если $V$ конечно мерно, то отображение $\phi$ является изоморфизмом.
\end{enumerate}
\end{claim}
\begin{proof}
(1) Так как $\phi$ линейно, нам достаточно показать, что у него нулевое ядро. Давайте расшифруем, что это значит. Пусть $v\in V$ такой, что $\phi_v = 0$. Это значит, что $\phi_v \colon V^* \to F$ является нулевым отображением. То есть $\phi_v(\xi) = 0$ для любого $\xi \in V^*$. То есть $\xi(v) = 0$ для любого $\xi \in V^*$. Чтобы доказать, что $v = 0$, нам достаточно показать, что если $v\neq0$, то найдется функционал $\xi \in V^*$ такой, что $\xi(v)\neq 0$. Действительно, если $v\neq 0$, то $\{v\}$ -- линейно независимая система. Тогда ее можно дополнить до базиса $\{v\}\cup E$.\footnote{В конечномерном случае мы это умеем делать, а в бесконечномерном случае -- это аксиома эквивалентная аксиоме выбора.} Тогда положим отображение $\xi \colon V\to F$ по правилу $v\mapsto 1$, а $E$ идет целиком в $0$.\footnote{В конечномерном случае это утверждение~\ref{claim::Basis}. Для бесконечномерного случая можете проверить, что оно остается верным.} Тогда $\xi\in V^*$ и $\xi(v)\neq 0$, что и требовалось.

(2) Так как $V$ вкладывается в $V^{**}$, то нам достаточно убедиться, что они имеют одинаковую размерность. А это следует из утверждения~\ref{claim:DualBasis}, так как $\dim V = \dim V^* = \dim V^{**}$.
\end{proof}

Я конечно же постарался привести доказательство в бесконечно мерном случае. Однако, если все эти бесконечномерные пакости так претят вашей ранимой тонкой душевной организации, можете доказывать это утверждение только для конечно мерных пространств.

\paragraph{Замечания}
\begin{itemize}
\item Если мы стартовали с векторного пространства $V$, то можем начать строить цепочку пространств вида $V, V^*, V^{**}, \ldots$. Утверждение~\ref{claim:DualBasis} гласит, что с абстрактной точки зрения, все эти пространства одинаковые -- изоморфны и мы ничего не получили нового.

\item Однако, этот абстрактный изоморфизм ничего не знает про <<семантику>> наших пространств, а именно он ничего не знает про операцию применения функционалов к векторам. Изоморфизм $\phi\colon V\to V^{**}$ из утверждения~\ref{claim::DoubleDuoIsom} согласован с этой семантикой следующим образом. На паре пространств $V^*$, $V$ есть операция $V^*\times V\to F$ вычисления функционала и на паре пространств $V^{**}$, $V^*$ есть операция $V^{**}\times V^*\to F$ вычисления функционала. Рассмотрим следующую картинку
\begin{center}
\begin{tabular}{c|c|c}

{$V$}&{$V^*$}&{$V^{**}$}\\

\hline

{$v$}&{$\xi$}&{$\phi_v$}\\

\end{tabular}
\end{center}
Тогда мы можем применить $\xi$ к $v$ и получим $\xi(v)$, можем применить $\phi_v$ к $\xi$ и получим $\phi_v(\xi) = \xi(v)$. То есть рассматривать пару $(v,\xi)$ можно как пару $(\text{вектор}, \text{функция})$, а можно рассматривать как пару $(\text{функция}, \text{вектор})$. При этом операция вычисления функции на векторе будет одной и той же и задается правилом $\xi v$ (вот видите, как полезна симметричная запись).\footnote{Таким образом нам необходимо знать только про векторы и функционалы. Кроме того, оказывается, что если определить такую операцию над векторными пространствами как тензорное произведение, то все, что только можно определить в линейной алгебре, можно выразить через векторы и функционалы с помощью тензорного произведения, например, линейные отображения, операторы, билинейные формы (которые будут чуть позже) и много других объектов. Это все ведет к некоторому единому удобному тензорному языку.}


\item Теперь еще раз рассмотрим последовательность $V, V^*, V^{**}$. Если мы выберем в $V$ некоторый базис $e_1,\ldots,e_n$, то в $V^*$ можем найти к нему двойственный $\xi_1,\ldots,\xi_n$. После чего найдем к последнему двойственный $\eta_1,\ldots,\eta_n$ в $V^{**}$. Но теперь вспомним, что $V^{**}$ изоморфно $V$. А так как изоморфизм между $V$ и $V^{**}$ согласован с применением функционалов, то базис $\eta_1,\ldots,\eta_n$ перейдет в $e_1,\ldots,e_n$.

\item Чуть ниже я покажу более аккуратно, что значит, что изоморфизм $\phi\colon V\to V^{**}$ в некотором смысле канонический, а между $V$ и $V^*$ таких не бывает.
\end{itemize}


\subsection{Сопряженное линейное отображение}

Выше мы показали, что в конечномерном случае все три пространства $V$, $V^*$ и $V^{**}$ изоморфны между собой. Однако, в случае $V$ и $V^{**}$ мы не просто показали это, мы построили некоторый замечательнейший изоморфизм между ними. Ниже очень хочется объяснить, а что же такого замечательного в этом изоморфизме и почему подобного замечательного изоморфизма нет между $V$ и $V^*$. Оказывается, что изоморфизм $\phi\colon V \to V^{**}$ в некотором смысле согласован с линейными отображениями, а подобного согласованного изоморфизма между $V$ и $V^*$ просто не бывает. Для того, чтобы объяснить, что все это значит, мне нужно для начала определить сопряженное линейное отображение.

Пусть $\varphi\colon V\to U$ -- некоторое линейное отображение. Мы хотим определить другое линейное отображение $\varphi^* \colon U^*\to V^*$ следующим образом:
\[
\xymatrix@R=6pt{
	{V}\ar[r]^{\varphi}\ar@{-->}@/^16pt/[rr]^{\xi\circ \varphi}&{U}\ar[r]^{\xi}&{F}\\
	{V^*}&{U^*}\ar[l]_{\phi^*}&{}\\
	{\xi\circ \varphi}&{\xi}\ar@{|->}[l]&{}\\
}
\]
Давайте поясним, что нарисовано на диаграмме. Нам надо определить $\varphi^*\colon U^*\to V^*$. То есть для любого $\xi\in U^*$ нам надо задать $\varphi^*(\xi)\in V^*$. Последнее означает, что по линейному функционалу на $U$, нам надо как-то построить линейный функционал на $V$. Предлагается определить $\varphi^*(\xi) = \xi \varphi$ как композицию.

\begin{definition}
Для линейного отображения $\varphi\colon V\to U$ сопряженным (или двойственным) линейным отображением называется $\varphi^*\colon U^*\to V^*$ по правилу $\xi \mapsto \varphi^*(\xi) = \xi \varphi$.
\end{definition}

\paragraph{Функториальность звездочки}

Обратите внимание, что мы теперь построили очень любопытный математический агрегат. А именно, пусть у нас есть мешок всех векторных пространств и линейных отображений между ними. Тогда для каждого векторного пространства $V$ мы можем построить новое векторное пространство $V^*$. Кроме того, мы умеем действовать не только на векторных пространствах, но и на отображениях между ними. То есть каждому отображению $\varphi\colon V\to U$ мы ставим в соответствие $\varphi^*\colon U^*\to V^*$. Про это надо думать так: все векторные пространства образуют (охренительно огроменнейший) граф, у которого вершины -- векторные пространства, а ребра -- линейные отображения. Мы построили отображение из этого графа в себя, которое разворачивает стрелки. Кроме того, это отображение согласовано с композицией в следующем смысле:\footnote{Я не хочу вводить формальные определения, но мы реально только что построили контровариантный функтор из категории векторных пространств в себя, что бы это ни значило.}
\begin{itemize}
\item Если $\Identity\colon V\to V$, тогда $\Identity^*\colon V^*\to V^*$ является тождественным отображением, то есть $\Identity^* = \Identity$.

\item Если $\varphi\colon V\to U$ и $\psi\colon U\to W$, тогда $(\psi\varphi)^* = \varphi^*\psi^*$, то есть звездочка меняет местами порядок отображений. Графически это можно изобразить так:
\[
\text{Если коммутативна диаграмма }
\xymatrix@R=6pt{
	{V}\ar[r]^-{\varphi}\ar@{-->}@/^16pt/[rr]^{\psi\circ \varphi}&{U}\ar[r]^-{\psi}&{W,}\\
}
\text{ то коммутативна диаграмма }
\xymatrix@R=6pt{
	{V^*}&{U^*}\ar[l]_-{\varphi^*}&{W^*}\ar[l]_-{\psi^*}\ar@{-->}@/_16pt/[ll]_{(\psi\circ \varphi)^*}\\
}
\]
Коммутативность диаграммы, означает, что любые два пути ведущие из одной вершины в другую приводят к одному результату.

%% Надо добавить свойства линейности!!!!

\end{itemize}


\paragraph{Матрица сопряженного линейного отображения}

\begin{claim}\label{claim::DualHomMatrix}
Пусть $V$ -- векторное пространство с базисом $e_1,\ldots,e_n$, $U$ -- векторное пространство с базисом $f_1,\ldots,f_m$. Пусть $e^1,\ldots,e^n$ -- двойственный базис в $V^*$ и $f^1,\ldots,f^m$ -- двойственный базис в $U^*$. Пусть $\varphi \colon V\to U$ -- некоторый линейное отображение с матрицей $A_\varphi$ в базисах $e_i$ и $f_j$ и пусть $\varphi^*\colon U^*\to V^*$ -- сопряженное линейное отображение с матрицей $A_{\varphi^*}$ в базисах $f^i$ и $e^j$. Тогда $A_{\varphi^*} = A_{\varphi}^t$.
\end{claim}
\begin{proof}
По определению матрицы линейного отображения
\[
\varphi(e_1,\ldots,e_n) = (f_1,\ldots,f_m)A_{\varphi}\quad\text{и}\quad
\varphi^*(f^1,\ldots,f^m) = (e^1,\ldots,e^n)A_{\varphi^*}
\]
Кроме того, по определению двойственного базиса нам дано
\[
\begin{pmatrix}
{e^1}\\{\vdots}\\{e^n}
\end{pmatrix}
\begin{pmatrix}
{e_1}&{\ldots}&{e_n}
\end{pmatrix}
=E\quad\text{и}\quad
\begin{pmatrix}
{f^1}\\{\vdots}\\{f^m}
\end{pmatrix}
\begin{pmatrix}
{f_1}&{\ldots}&{f_m}
\end{pmatrix}
=E
\]
В начале распишем следующее равенство
\[
(f^1\varphi,\ldots,f^m\varphi)=\varphi^*(f^1,\ldots,f^m) = (e^1,\ldots,e^n)A_{\varphi^*}
\]
Теперь транспонируем его и получим
\[
\begin{pmatrix}
{f^1 \varphi}\\{\vdots}\\{f^m\varphi}
\end{pmatrix}
=
\begin{pmatrix}
{f^1}\\{\vdots}\\{f^m}
\end{pmatrix} \varphi
=
A_{\varphi^*}^t
\begin{pmatrix}
{e^1}\\{\vdots}\\{e^n}
\end{pmatrix}
\]
Умножим левую и праву часть полученного равенства на $(e_1,\ldots,e_n)$, получим
\begin{gather*}
\begin{pmatrix}
{f^1}\\{\vdots}\\{f^m}
\end{pmatrix} 
\varphi (e_1,\ldots,e_n)
=
A_{\varphi^*}^t
\begin{pmatrix}
{e^1}\\{\vdots}\\{e^n}
\end{pmatrix}
\begin{pmatrix}
{e_1}&{\ldots}&{e_n}
\end{pmatrix}\\
\begin{pmatrix}
{f^1}\\{\vdots}\\{f^m}
\end{pmatrix} 
(f_1,\ldots,f_m)A_{\varphi}
=
A_{\varphi^*}^t\\
A_{\varphi} = A_{\varphi^*}^t
\end{gather*}
что и требовалось.
% TO DO
% Добавить координатное доказательство, оно может быть проще и понятнее
\end{proof}

\paragraph{Замечание}

То есть транспонирование матриц имеет следующий философский смысл: это переход к сопряженному линейному отображению в двойственном пространстве. Заметили, что транспонирование и звездочка меняют местами порядок отображений? Последнее утверждение показывает, что это не случайное совпадение.

\paragraph{Канонический изоморфизм}

\begin{claim}\label{claim::CanonicalIsomorphism}
Пусть $\varphi\colon V\to U$ -- линейное отображение и $\phi\colon V\to V^{**}$ и $\phi\colon U\to U^{**}$ -- канонические изоморфизмы на второе сопряженное. В этом случае коммутативна следующая диаграмма:
\[
\xymatrix{
	{V}\ar[r]^{\varphi}\ar[d]^{\phi}&{U}\ar[d]^{\phi}\\
	{V^{**}}\ar[r]^{\varphi^{**}}&{U^{**}}
}
\]
то есть $\phi \varphi = \varphi^{**}\phi$.
\end{claim}
\begin{proof}
Давайте распишем, как вектор $v\in V$ двигается по этой диаграмме
\[
\xymatrix{
	{v}\ar@{|->}[rr]\ar@{|->}[d]&{}&{\varphi(v)}\ar@{|->}[d]\\
	{\phi_v}\ar@{|->}[r]&{\phi_v\circ \varphi^*}\ar@{=}[r]^{?}&{\phi_{\varphi(v)}}
}
\]
И нам надо проверить равенство с вопросиком. Для этого надо сравнить действие левой и правой части на произвольном $\xi\in U^*$. Получим
\[
(\phi_v\circ \varphi^*)(\xi) = \phi_v(\varphi^*(\xi)) = \phi_v(\xi\circ\varphi) = (\xi\circ \varphi) (v) = \xi(\varphi(v)) = \phi_{\varphi(v)}(\xi)
\]
что и требовалось.
\end{proof}

\paragraph{Замечания}
\begin{itemize}
\item Последнее утверждение показывает, что изоморфизмы $\phi\colon V\to V^{**}$ согласованы с линейными отображениями в том смысле, что коммутативны некоторые диаграммы. Как надо думать про это? Смысл $\varphi$ в том, что мы можем считать, что $V$ и $V^{**}$ -- это одно и то же. Согласованность с отображениями означает, что при таком отождествлении $V$ с $V^{**}$ и $U$ с $U^{**}$ отображение $\varphi$ превращается в $\varphi^{**}$.

\item С другой стороны, если мы захотим потребовать нечто подобное для $V$ и $V^*$ то мы получим диаграмму вида
\[
\xymatrix{
	{V}\ar[r]^{\varphi}\ar[d]^{\phi}&{U}\ar[d]^{\phi}\\
	{V^*}&{U^*}\ar[l]^{\varphi^*}
}
\]
Причем эта диаграмма должна быть коммутативной (то есть $\phi = \varphi^*\phi\varphi$) для всех отображений $\varphi\colon V\to U$, а вертикальные стрелки $\phi$ все должны быть изоморфизмами. Но выберем тогда в качестве $\varphi$ нулевое отображение и получим, что $\phi = 0$, противоречие.
\end{itemize}


\newpage
\section{Тензоры}

Всем хороши векторные пространства. Их элементы можно складывать и умножать на числа. Одна беда -- векторы нельзя перемножать. А очень хочется. А мы знаем, что когда нельзя, но очень хочется, то можно, а скорее даже нужно. Наша следующая задача научиться перемножать векторы. Окажется, что способов перемножать векторы много и результаты этих перемножений могут лежать где угодно. Однако, среди всех этих способов можно найти самую лучшую операцию перемножения и для нее построить особое пространство, в котором будут лежать произведения векторов. Вот это новое замечательное пространство будет называться тензорным произведением исходных векторных пространств. Чтобы начать разговор о тензорных произведениях, надо немного поговорить об операциях умножения или как более прилично говорить в математике -- билинейных отображениях.

\subsection{Билинейные отображения}

\begin{definition}
Пусть $V$, $U$ и $T$ -- векторные пространства над полем $F$. Отображение $\mu\colon V\times U \to T$ называется билинейным, если выполняется следующий набор аксиом
\begin{enumerate}
\item Для любых векторов $v_1,v_2\in V$ и $u\in U$ выполнено
\[
\mu(v_1 + v_2, u) = \mu(v_1, u) + \mu(v_2, u)
\]
\item Для любых векторов $v\in V$, $u\in U$ и чисел $\lambda \in F$ выполнено
\[
\mu(\lambda v, u) = \lambda \mu(v, u)
\]
\item Для любых векторов $v\in V$ и $u_1,u_2\in U$ выполнено
\[
\mu(v, u_1 + u_2) = \mu(v, u_1) + \mu(v, u_2)
\]
\item Для любых векторов $v\in V$, $u\in U$ и чисел $\lambda \in F$ выполнено
\[
\mu(v, \lambda u) = \lambda \mu(v, u)
\]
\end{enumerate}
\end{definition}

Обратите внимание, что отображение $\mu$ -- это функция двух аргументов. А у любой функции двух аргументов есть операторная запись.\footnote{Операторная и функциональная запись функций с двумя аргументами многим должна быть знакома по языкам программирования.} Например, на запись $\mu(v, u) $ можно смотреть как на $v * u$. В этом случае определение выше можно переписать в эквивалентной форме, которая психологически кому-то может быть более близка.
\begin{definition}
Пусть $V$, $U$ и $T$ -- векторные пространства над полем $F$. Отображение $*\colon V\times U \to T$ называется билинейным (или умножением), если выполняется следующий набор аксиом
\begin{enumerate}
\item Для любых векторов $v_1,v_2\in V$ и $u\in U$ выполнено
\[
(v_1 + v_2) * u = v_1 * u + v_2 * u
\]
\item Для любых векторов $v\in V$, $u\in U$ и чисел $\lambda \in F$ выполнено
\[
(\lambda v)*u = \lambda (v * u)
\]
\item Для любых векторов $v\in V$ и $u_1,u_2\in U$ выполнено
\[
v * (u_1 + u_2) = v * u_1 + v * u_2
\]
\item Для любых векторов $v\in V$, $u\in U$ и чисел $\lambda \in F$ выполнено
\[
v *(\lambda u) = \lambda (v * u)
\]
\end{enumerate}
\end{definition}
То есть билинейная операция -- это умножение, от которого мы лишь требуем дистрибутивности относительно сложения и согласованность с умножением на скаляр.\footnote{Когда вы чуть-чуть дорастете до курса алгебры, вы узнаете, что эти свойства определяют любую операцию умножения в алгебраических структурах.}

\paragraph{Примеры}
Давайте сделаем передышку и взглянем на несколько полезных примеров:
\begin{itemize}
\item Пусть $V$ и $U$ -- два произвольных векторных пространства. Зададим операцию умножения следующим образом $\mu\colon V\times U \to 0$ по правилу $(v, u) \mapsto 0$. Это самая дурацкая операция умножения, которую только можно придумать. Она не несет в себе никакой полезной информации и именно в этом заключается ее польза.

\item Пусть $V = F^n$ и $U = F^n$ зададим операцию умножения так
\[
F^n \times F^n \to F,\quad (x, y) \mapsto x^t y
\]

\item Пусть $V = F^n$ и $U = F^n$ зададим операцию умножения так
\[
F^n \times F^n \to \operatorname{M}_{n}(F),\quad (x, y)\mapsto xy^t
\]

\item Пусть $V = U= \operatorname{M}_n(F)$, тогда рассмотрим операцию матричного умножения
\[
\operatorname{M}_n(F)\times \operatorname{M}_n(F)\to \operatorname{M}_n(F),\quad (A, B)\mapsto AB
\]

\item Пусть $\mu \colon V\times U\to T$ -- произвольная операция умножения и пусть $\phi \colon T\to E$ -- произвольное линейное отображение векторных пространств. Тогда композиция $\phi\circ \mu \colon V\times U \to E$ будет так же операцией умножения.


Таким образом некоторые операции умножения можно выразить через другие с помощью линейных отображений. В таком смысле операция $\mu$ интереснее операции $\phi \circ \mu$, так как вторая получается из первой. Идея заключается в том, что если мы будем знать все про первую операцию, то мы будем знать все и про вторую, так как она выражается через первую.
\end{itemize}

Прежде всего мы видим, что бывает много разных операций умножения на одной и той же паре пространств. Кроме того, в силу последнего замечания самый большой интерес для нас представляет операция умножения, через которую можно выразить все другие операции (при условии, что такая существует). Оказывается, что подобная замечательная операция существует и в некотором смысле единственная. Но начнем мы с формулировки универсального свойтства.

\begin{definition}
Пусть $V$, $U$, $T$ -- векторные пространства и $\mu \colon V\times U \to T$ -- операция умножения. Тогда $\mu$ называется универсальной, если выполняется следующее свойство: для любого векторного пространства $E$ и любой операции умножения $\phi \colon V\times U\to E$ найдется единственное линейное отображение $\psi\colon T\to E$ такое, что $\phi = \psi \circ \mu$. Продемонстрируем происходящее на диаграмме ниже
\[
\xymatrix{
	{V\times U}\ar[rr]^{\mu}\ar[rd]_{\phi}&{}&{T}\ar@{-->}[dl]^{\psi}\\
	{}&{E}&{}
}
\]
\end{definition}
Сделаю одно полезное замечание. Если умножение $\mu\colon V\times U\to T$ универсальное, это значит, что любая операция умножения $\phi \colon V\times U \to E$ соответствует единственному линейному отображению $\psi\colon T\to E$. Таким образом мы можем свести изучение билинейных отображений к линейным. Подобная процедура позволит избежать построения теории билинейных отображений, аналогичной теории линейных отображений, и пользоваться уже готовым аппаратом.

\begin{claim}
Пусть $V$, $U$ и $T$ -- векторные пространства над полем $F$ и пусть $\mu \colon V\times U\to T$ -- универсальное умножение. Тогда
\begin{enumerate}
\item Если $\phi \colon T\to T$ -- линейное отображение такое, что $\mu = \phi \circ \mu$, то есть следующая диаграмма коммутативна:
\[
\xymatrix@R=10pt{
	{}&{}&{T}\ar[dd]^{\phi}\\
	{V\times U}\ar[urr]^{\mu}\ar[drr]_{\mu}&{}&{}\\
	{}&{}&{T}\\
}
\]
То $\phi = \Identity_T$, то есть $\phi$ обязательно тождественное отображение.

\item Пусть $\mu'\colon V\times U\to T'$ -- другая универсальная операция умножения. Тогда существует единственное линейное отображение $\phi\colon T\to T'$ такое, что $\mu' = \phi \circ \mu$ и при этом $\phi$ обязательно окажется изоморфизмом. На диаграмме это можно изобразить так
\[
\xymatrix@R=10pt{
	{}&{}&{T}\ar@{-->}[dd]^{\phi}\\
	{V\times U}\ar[urr]^{\mu}\ar[drr]_{\mu'}&{}&{}\\
	{}&{}&{T'}\\
}
\]
\end{enumerate}
\end{claim}
\begin{proof}
1) С одной стороны у нас есть отображение $\phi\colon T\to T$ такое, что $\mu = \phi \circ \mu$. С другой стороны отображение $\Identity_T\colon T\to T$ тоже удовлетворяет условию $\mu = \Identity_T \circ \mu$. Изобразим это на диаграмме ниже
\[
\xymatrix@R=10pt{
	{}&{}&{T}\ar@<3pt>[dd]^{\phi}\ar@<-3pt>[dd]_{\Identity_T}\\
	{V\times U}\ar[urr]^{\mu}\ar[drr]_{\mu}&{}&{}\\
	{}&{}&{T}\\
}
\]
Но так как $\mu$ было универсальным, то существует одно единственное отображение с таким свойством. А значит $\phi = \Identity_T$.

2) Пусть теперь у нас есть две универсальные операции умножения.
\[
\xymatrix@R=10pt{
	{}&{}&{T}\\
	{V\times U}\ar[urr]^{\mu}\ar[drr]_{\mu'}&{}&{}\\
	{}&{}&{T'}\\
}
\]
Так как $\mu$ является универсальной операцией, то существует единственное линейное отображение $\phi\colon T\to T'$ такое, что $\mu' = \phi \circ \mu$. В терминах диаграммы
\[
\xymatrix@R=10pt{
	{}&{}&{T}\ar@{-->}[dd]^{\phi}\\
	{V\times U}\ar[urr]^{\mu}\ar[drr]_{\mu'}&{}&{}\\
	{}&{}&{T'}\\
}
\]
С другой стороны, так как $\mu'$ является универсальной операцией умножения, существует единственное линейное отображение $\phi'\colon T'\to T$ такое, что $\mu = \phi' \circ \mu'$. На диаграмме это изображается так
\[
\xymatrix@R=10pt{
	{}&{}&{T}\ar@{-->}@<3pt>[dd]^{\phi}\\
	{V\times U}\ar[urr]^{\mu}\ar[drr]_{\mu'}&{}&{}\\
	{}&{}&{T'}\ar@{-->}@<3pt>[uu]^{\phi'}\\
}
\]
Тогда композиция $\phi'\circ \phi \colon T\to T$ удовлетворяет условиям первого пункта утверждения, а значит равна тождественному отображению на $T$, то есть $\phi'\circ \phi = \Identity_T$. Аналогично, отображение $\phi\circ \phi'\colon T'\to T'$ удовлетворяет условиям первого пункта утверждения, а значит $\phi\circ \phi' = \Identity_{T'}$. Таким образом $\phi$ и $\phi'$ -- это два взаимнообратных изоморфизма, что и требовалось.
\end{proof}

Последнее утверждение означает, что любые две универсальные операции одинаковые. Одинаковость понимается так: мы можем найти изоморфизм между их областями значений ($T$ и $T'$ в обозначениях предыдущего утверждения) такой, что под действием этого изоморфизма одна операция превращается в другую. Потому нет никакого смысла отличать одну универсальную операцию от другой. Все свойства у этих универсальных операций автоматически будут одинаковые.

\ProvidesFile{lecture17.tex}[Лекция 17]


\begin{claim}\label{claim::IdealRootDec}
Пусть $\varphi\colon V\to V$ -- линейный оператор над полем $F$, $f\in F[t]$ зануляющий многочлен такой, что $f = p q \in F[t]$, где $(p, q) = 1$. Тогда
\begin{enumerate}
\item $\ker p(\varphi) = \Im q(\varphi)$.

\item $\ker q(\varphi) = \Im p(\varphi)$.

\item $V = \ker p(\varphi) \oplus \ker q(\varphi)$.

\item Для любого $\varphi$ инвариантного подпространства $U\subseteq V$ выполнены равенства
\[
U = U\cap \ker p(\varphi) \oplus U \cap \ker q(\varphi) = \pi_1(U)\oplus \pi_2(U)
\]
где $\pi_1 \colon V\to V$ -- проектор на $\ker p(\varphi)$ вдоль $\ker q(\varphi)$, то есть для вектора $v = u + w$, где $u\in \ker p(\varphi)$ и $w\in \ker q(\varphi)$, имеем $\pi_1(v) = u$. Аналогично, $\pi_2\colon V\to V$ -- проектор на $\ker q(\varphi)$ вдоль $\ker p(\varphi)$.

\item Если $f$ -- минимальный многочлен для $\varphi$, то  многочлен $p$ будет минимальным для оператора $\varphi|_{\ker p(\varphi)}$.

\end{enumerate}
\end{claim}
\begin{proof}
В начале сделаем некие общие подготовительные работы. Из того, что многочлены $p$ и $q$ взаимнопросты, по расширенному алгоритму Евклида, найдутся многочлены $u,v\in F[t]$ такие, что $1 = u(t)p(t) + v(t)q(t)$. Подставим в это равенство и в $f$ оператор $\varphi$, получим два равенства
\begin{gather*}
\Identity = u(\varphi) p(\varphi) + v(\varphi) q(\varphi)\\
0 = p(\varphi) q(\varphi)
\end{gather*}

(1) и (2). Так как утверждения~(1) и~(2) симметричны, то достаточно доказать одно из них. 

В начале покажем, что $\Im q(\varphi)\subseteq \ker p(\varphi)$. Так как $p(\varphi)q(\varphi) = 0$, то для любого $w\in V$ верно, что $p(\varphi)q(\varphi)w = 0$, но это означает, что $q(\varphi)w \in\ker p(\varphi)$, то есть $\Im q(\varphi)\subseteq \ker p(\varphi)$.

Наоборот, возьмем $w\in \ker p(\varphi)$ и применим к нему первое операторное равенство, получим
\[
w = u(\varphi) p(\varphi) w + v(\varphi) q(\varphi) w = v(\varphi) q(\varphi) w =   q(\varphi) v(\varphi) w\in \Im q(\varphi)
\]

(3) Из взаимной простоты $p$ и $q$ следует, что $\ker p(\varphi) \cap \ker q(\varphi) = 0$ по утверждению~\ref{claim::CoprimeKernels}. Значит сумма этих под пространств прямая, то есть $\ker p(\varphi) \oplus \ker q(\varphi) \subseteq V$. Чтобы показать равенство, возьмем произвольный $w\in V$ и рассмотрим
\[
w = u(\varphi) p(\varphi) w + v(\varphi) q(\varphi) w = p(\varphi) u(\varphi) w + q(\varphi) v(\varphi) w\in \Im p(\varphi) +\Im q(\varphi) = \ker q(\varphi) + \ker p(\varphi)
\]
Здесь последнее равенство выполнено в силу предыдущих двух пунктов.

(4) Пусть $U\subseteq V$ инвариантное подпространство. Тогда $U\cap \ker p(\varphi) \subseteq U$ и $U\cap \ker q(\varphi) \subseteq U$, а значит и их сумма лежит в $U$. С другой стороны, так как $\ker p(\varphi)$ и $\ker q(\varphi)$ пересекаются по нулю, то и подпространства $U\cap \ker p(\varphi)$ и $U\cap \ker q(\varphi)$ тоже пересекаются по нулю. А значит они образуют прямую сумму внутри $U$, то есть $U \supseteq U\cap \ker p(\varphi) \oplus U \cap \ker q(\varphi)$. Осталось доказать, обратное включение. Пусть $w\in U$, тогда 
\[
w = v(\varphi) q(\varphi) w + u(\varphi) p(\varphi) w
\]
Так как $U$ было инвариантным и  $w\in U$, то $v(\varphi) q(\varphi) w \in U$. С другой стороны,
\[
p(\varphi) v(\varphi) q(\varphi) w = v(\varphi) p(\varphi)  q(\varphi) w = 0
\]
А значит, $v(\varphi) q(\varphi) w\in \ker p(\varphi)$. Таким образом, мы проверили, что первое слагаемое лежит в $U\cap \ker p(\varphi)$. Аналогично проверяется включение для второго слагаемого.

Теперь мы знаем, что любое инвариантное $U\subseteq V$ имеет вид $U = U_1 \oplus U_2$, где $U_1 = U\cap \ker p(\varphi)$ и $U_2 = U\cap \ker q(\varphi)$. Давайте применим $\pi_1$ к $U$. По определению 
\[
\pi_1(U) = \pi_1(U_1 + U_2) = \pi_1(U_1) = U_1.
\]
Аналогично $\pi_2 (U) = U_2$. А значит $U = U_1\oplus U_2 = \pi_1(U) \oplus \pi_2(U)$, то есть мы доказали второе равенство.


(5) Теперь мы считаем, что $f = f_\text{min}$ для $\varphi$ на $V$. Заметим, что
\[
p(\varphi|_{\ker p(\varphi)}) = p(\varphi)|_{\ker p(\varphi)} = 0
\]
Значит $p$ зануляет $\varphi|_{\ker p(\varphi)}$. Так как минимальный многочлен обязательно делит $p$, то нам надо показать, что никакой делитель $p$ отличный от $p$ не зануляет $\varphi|_{\ker p(\varphi)}$. Предположим противное, пусть $p_0 | p$ и $p_0(\varphi|_{\ker p(\varphi)}) = 0$. Тогда рассмотрим многочлен $g = p_0 q$ и покажем, что $g(\varphi) = 0$. Так как $V = \ker p(\varphi) \oplus \ker q(\varphi)$, то нам достаточно показать, что  $g(\varphi)$ действует нулем на любом векторе из $\ker p(\varphi)$ и на любом векторе из $\ker q(\varphi)$. Но на $\ker q(\varphi)$ нулем действует $q(\varphi)$, а $g(\varphi) = p_0(\varphi)q(\varphi)$. А на $\ker p(\varphi)$ оператор $p_0(\varphi)$ действует нулем по выбору $p_0$, а значит и $g(\varphi) = q(\varphi) p_0(\varphi)$ действует нулем. То есть многочлен $g$ зануляет $\varphi$ и имеет степень меньше, чем $f_\text{min}$, противоречие.

\end{proof}

\begin{claim}\label{claim::GenRootDec}
Пусть $\varphi\colon V\to V$ -- линейный оператор, $f_\text{min}$ -- его минимальный многочлен. Пусть
\[
f_\text{min} = p_1^{k_1} \ldots p_r^{k_r}
\]
разложение минимального в неприводимые многочлены. Тогда
\begin{enumerate}
\item  $V^{p_i} = \ker p_i^{k_i}$ причем $k_i$ -- минимальное такое $k$ для которого выполнено равенство $V^{p_i} = \ker p_i^{k}$.

\item $V = V^{p_1}\oplus \ldots \oplus V^{p_r}$.

\item Любое инвариантное подпространство $U\subseteq V$ имеет вид $U = U_1\oplus \ldots \oplus U_r$, где $U_i \subseteq V^{p_i}$ -- произвольные инвариантные подпространства.
\end{enumerate}
\end{claim}
\begin{proof}
1) Докажем утверждение для $i = 1$. Тогда сгруппируем множители минимального многочлена следующим образом $f_\text{min} = p_1^{k_1} g$. Так как минимальный многочлен -- зануляющий, то $p_1(\varphi)^{k_1}g(\varphi) = 0$ на $V$, а значит и на $V^{p_1}$. С другой стороны, по лемме о стабилизации, мы знаем, что $V^{p_1} = \ker p_1(\varphi)^{d}$ для некоторого $d$. И пусть $d$ выбрано минимальным. Тогда утверждение~\ref{claim::CoprimeKernels}, примененное для многочленов $p_1^d$ и $g$, означает, что оператор $g(\varphi)$ будет обратим на пространстве $V^{p_1}$. А значит можно в равенстве $p_1(\varphi)^{k_1}g(\varphi) = 0$  сократить на $g(\varphi)$ и получить, что $p_1(\varphi)^{k_1} = 0$ на $V^{p_1}$. То есть $V^{p_1}\subseteq \ker p_1(\varphi)^{k_1}$. Значит $d \leqslant k_1$.

С другой стороны, по утверждению~\ref{claim::IdealRootDec} пункт~(5) $p_1^{k_1}$ будет минимальным многочленом для $\varphi$ ограниченного на $\ker p_1(\varphi)^{k_1} = V^{p_1}$. А значит, никакая меньшая степень $p_1(\varphi)^k$ не зануляется на $V^{p_1}$. А значит $d \geqslant k_1$.

2) Сгруппируем множители $f_\text{min}$ следующим образом
\[
f_\text{min} = \underbrace{p_1^{k_1}}_{p}\underbrace{\ldots p_r^{k_r}}_{q}
\]
Тогда, $p$ и $q$ взаимно просты и по утверждению~\ref{claim::IdealRootDec} пункт~(3) мы можем разложить $V$ в прямую сумму 
\[
V = \ker p(\varphi) \oplus \ker q(\varphi) = \ker p_1^{k_1}(\varphi) \oplus V_1
\]
где $V_1 = \ker q(\varphi)$. При этом минимальный многочлен оператора $\varphi_1 = \varphi|_{\ker q(\varphi)}$ есть $q$ (утверждение~\ref{claim::CoprimeKernels}). Тогда продолжая по индукции для оператора $\varphi_1$ в пространстве $V_1$, мы можем считать, что
\[
V = \ker p_1^{k_1}(\varphi)\oplus \ldots \oplus \ker p_r^{k_r}(\varphi)
\]
Теперь осталось показать, что $\ker p_i^{k_i}(\varphi) =  V^{p_i}$, но это следует из предыдущего пункта.

3) Это непосредственно следует из утверждения~\ref{claim::IdealRootDec} пункт~(4) индукцией по количеству прямых слагаемых.
\end{proof}




\subsection{Геометрический смысл кратности корней минимального и характеристического многочлена}

\begin{claim}\label{claim::RootMultGeom}
Пусть $\varphi\colon V \to V$ -- линейный оператор и $\lambda \in \spec_F \varphi$. Пусть число $k$ выбрано так, что
\[
\ker (\varphi-\lambda\Identity)^{k-1} \neq \ker (\varphi-\lambda\Identity)^k = \ker (\varphi-\lambda\Identity)^{k+1}
\]
Тогда
\begin{enumerate}
\item Число $k$ -- это кратность $\lambda$ в минимальном многочлене оператора $\varphi$.

\item Число $\dim \ker(\varphi - \lambda \Identity)^k = \dim V^\lambda$ -- это кратность корня $\lambda$ в характеристическом многочлене оператора $\varphi$.
\end{enumerate}
\end{claim}
\begin{proof}
(1) Это частный случай утверждения~\ref{claim::GenRootDec} пункт~(1) для $p_1(x) = x-\lambda$.

(2) Пусть $f_\text{min} = (x-\lambda)^k g(x)$ -- минимальный многочлен для $\varphi$ на $V$. Тогда пользуясь утверждением~\ref{claim::IdealRootDec} пункт~(3) и предыдущим пунктом этого утверждения, мы видим, что $V = V^\lambda \oplus \ker g(\varphi)$. Теперь давайте посчитаем характеристический многочлен для $\varphi$. По утверждению~\ref{claim::RestrictionChar} $\chi_\varphi(t)$ есть произведение $\chi_{\varphi|_{V^\lambda}}$ и $\chi_{\varphi|_{\ker g(\varphi)}}$. Утверждение~\ref{claim::CharPolyOnRootSpace} гласит, что $\chi_{\varphi|_{V^\lambda}}(t) = (t - \lambda)^{\dim V^\lambda}$. А утверждение~\ref{claim::CoprimeKernels} пункт~(2), что $\varphi - \lambda\Identity$ обратим на $\ker g(\varphi)$. Значит, оператор $\varphi|_{\ker g(\varphi)}$ не содержит $\lambda$ в своем спектре, а значит $\lambda$ не корень $\chi_{\varphi|_{\ker g(\varphi)}}$. Это завершает доказательство.
\end{proof}
